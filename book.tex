%
%
% RECOMMENDED IMPROVEMENTS
%
%
% -Gimbal lock. Link video from youtube that shows the effect. Add language that the X and Z axes align.
% - Quaternions. It can be shown that... Explain the Euler axis comes from the Eigenvector of the rotation matrix
% - Extend take home lessons of chapter kinematics by insights into Euler angles and quaternions
% - Equation 3.5 should be _0^2 T, not f(q)
% - Around Figure 3.3, solutions are given for \alpha and b, not for \alpha and \beta. 
% Link a video that shows how to get the Jacobian from DH matrices in Section 3.3.1 (see ecture notes)
% Add ellipsoid to linear algebra section http://ee263.stanford.edu/lectures/ellipsoids.pdf




%\documentclass[11pt]{scrbook} % use larger type; default would be 10pt

% WITH BLEED
% US Trade => 6x9, with a 0.125 bleed
% Adjust images size and gutter so tabs bleed by .125
% See https://www.createspace.com/Products/Book/InteriorPDF.jsp

%\documentclass[paper=6.14in:9.21in,pagesize=pdftex,11pt,twoside,openright]{scrbook}

\documentclass[paper=7in:9in,pagesize=pdftex,11pt,twoside,openright]{scrbook}


%openright
% Paper width
% W = 7.125in (7+0.125 --- bleed)
% Paper height
% H = 9.25in (9+2*.125 --- bleed)
% Paper gutter
% BCOR = 0.375in (0.5+0.5-0.625 --- margin with bleed)
% Margin (0.5in imposed on lulu, recommended on createspace)
% m = 0.625in (0.5+0.125 --- bleed)
% Text height
% h = H - 2m = 8in
% Text width
% w = W - 2m - BCOR = 4.5in
\areaset[0.375in]{5.5in}{8in}
\usepackage[utf8]{inputenc} % set input encoding (not needed with XeLaTeX)

% \usepackage{subfig} % make it possible to include more than one captioned figure/table in a single float
% Subfigure is deprecated
% Subfig is not compatible with hyperref
\usepackage{subcaption}
\captionsetup{format=plain,indention=.4cm,labelformat=simple, labelfont = sl, labelsep=period, font=small}
% \captionsetup[sub]{labelformat=simple,labelsep=period, font=small}

%%% Examples of Article customizations
% These packages are optional, depending whether you want the features they provide.
% See the LaTeX Companion or other references for full information.

\usepackage{mitpress}
\usepackage[usenames, dvipsnames, table]{xcolor}

\usepackage[framemethod=TikZ]{mdframed}
\mdfsetup{skipbelow=2pt, backgroundcolor=gray!8, linecolor=black!40}

\usepackage[top=1in, bottom=1in, left=1in, right=1in]{geometry}

\usepackage{graphicx} % support the \includegraphics command and options
%\usepackage{wrapfig}

% \usepackage[parfill]{parskip} % Activate to begin paragraphs with an empty line rather than an indent

%%% PACKAGES
\usepackage{booktabs} % for much better looking tables
\usepackage{array} % for better arrays (eg matrices) in maths
\usepackage{paralist} % very flexible & customisable lists (eg. enumerate/itemize, etc.)
\usepackage{verbatim} % adds environment for commenting out blocks of text & for better verbatim
% These packages are all incorporated in the memoir class to one degree or another...
\usepackage{amsmath}
\usepackage{mathtools}
\usepackage{amsfonts}
\usepackage{amssymb}
\usepackage{hyperref}
\usepackage{harvard}
\usepackage{fancyvrb}
\usepackage{marginnote}
\usepackage[capitalize, noabbrev]{cleveref} % Used to improve refernces
\usepackage{tikz}
\usetikzlibrary{automata, positioning}
\usepackage{tikz-qtree}
\usepackage{forest}
\usepackage{enumitem}
\usepackage[xindy]{imakeidx}

\colorlet{mylinkcolor}{BrickRed}
\colorlet{mycitecolor}{PineGreen}
\colorlet{myurlcolor}{NavyBlue}

\hypersetup{
  linkcolor  = mylinkcolor,
  citecolor  = mycitecolor,
  urlcolor   = myurlcolor,
  colorlinks = true,
}

\usepackage{makeidx}
\usepackage{idxlayout}

\usepackage{import}
\usepackage{xifthen}
\usepackage{pdfpages}
\usepackage{transparent}

% \let\comment=\relax
\usepackage[markup=bfit, deletedmarkup=sout, authormarkup=superscript, commandnameprefix=ifneeded]{changes}
\definechangesauthor[name={ar}, color=teal]{AR}
\definechangesauthor[name={nc}, color=green]{NC}
\definechangesauthor[name={bh}, color=blue]{BH}
\definechangesauthor[name={td}, color= red]{TODO}

\newcommand{\ar}[1]{\added[id=AR  ]{#1}}
\newcommand{\nc}[1]{\added[id=NC  ]{#1}}
\newcommand{\bh}[1]{\added[id=BH  ]{#1}}
\newcommand{\td}[1]{\added[id=TODO]{#1}}
%\newcommand{\cref}[1]{Section \ref{#1}}

\newcommand{\hjb}[1]{{\color{red}#1}}


% fixing a scrbook error when compiling the citations
\DeclareOldFontCommand{\bf}{\normalfont\bfseries}{\mathbf}

\DeclareUnicodeCharacter{00A0}{ }

%%% HEADERS & FOOTERS
%\usepackage{fancyhdr} % This should be set AFTER setting up the page geometry
%\pagestyle{fancy} % options: empty , plain , fancy
%\renewcommand{\headrulewidth}{0pt} % customise the layout...
%\lhead{}\chead{}\rhead{}
%\lfoot{}\cfoot{\thepage}\rfoot{}

%%% APPEARANCE OF SECTIONS
\usepackage[up,md,sf]{titlesec}

%%% APPEARANCE OF ToC (Table of Content)
%\usepackage[nottoc,notlof,notlot]{tocbibind} % Put the bibliography in the ToC
%\usepackage[titles,subfigure]{tocloft} % Alter the style of the Table of Contents

\newcommand{\screencast}[2]{
  \marginnote{\href{#1}{\includegraphics[trim=5 20 0 0, clip, width=1.8cm]{figs/youtube/#2}}}}


%%% END Article customizations

%%% The ``real'' document content comes below...
\makeindex

%\title{Introduction to Autonomous Robots}
%\author{Nikolaus Correll}
%\date{} % Activate to display a given date or no date (if empty),
         % otherwise the current date is printed

\allowdisplaybreaks

\begin{document}
%\maketitle


\thispagestyle{empty}
\begin{flushleft}
Nikolaus Correll, Bradley Hayes,\\ Christoffer Heckman, and Alessandro Roncone \\~\\
Introduction to Autonomous Robots:\\ Mechanisms, Sensors, Actuators, and Algorithms\\~\\
v3.0, \today\\
%Magellan Scientific\\
%ISBN-13: 978-1493773077
%ISBN-13: 978-0692700877
\end{flushleft}

\vfill

\begin{figure}[!h]
\includegraphics[width=1.2in]{figs/by-nc-nd}
\end{figure}

Copyright in this monograph has been licensed exclusively to The MIT Press, \url{http://mitpress.mit.edu}, which will be releasing the final version to the public in 2022. All inquiries regarding rights should be addressed to The MIT Press, Rights and Permissions Department.
Source code of this book is licensed under a Creative Commons Attribution-NonCommercial-NoDerivatives 4.0 International (CC BY-NC-ND 4.0). You are free to share, i.e., copy, distribute and transmit sources under the following conditions: you must attribute the work to its main author, you may not use this work for commercial purposes, and if you remix or modify this work you may not distribute the modified material. For more information, please consult \url{https://creativecommons.org/licenses/by-nc-nd/4.0/}.


\cleardoublepage
\thispagestyle{empty}
\topskip0pt
\vspace*{\fill}
\begin{center}
For Arthur, Tatiana, Benedict and Silvester\\
David
\\
Leonardo and Lily\\
future robot users\\
\end{center}
\vspace*{\fill}

\tableofcontents

\chapter*{Preface}

This book provides an algorithmic perspective to autonomous robotics to students with a sophomore-level of linear algebra and probability theory. Robotics is an emerging field at the intersection of mechanical engineering, electrical engineering, and computer science. With computers becoming more powerful, making robots smart is getting more and more into the focus of attention and robotics research most challenging frontier. While there is a large number of textbooks on the mechanics and dynamics of robots available to sophomore-level undergraduates, books that provide a broad algorithmic perspective are mostly limited to the graduate level. This book has therefore been developed not to create ``yet another textbook, but better than the others'', but to allow us to teach robotics to the 3rd and 4th year undergraduates at the Department of Computer Science at the University of Colorado.

Although falling under the umbrella of ``Artificial Intelligence'', standard AI techniques are not sufficient to tackle problems that involve uncertainty, such as a robot's interaction in the real world. This book uses simple trigonometry to develop the kinematic equations of manipulators and mobile robots, then introduces path planning, sensing, and lastly uncertainty. The robot localization problem is introduced by formally defining error propagation, which leads to Markov localization, Particle filtering and finally the Extended Kalman Filter, and Simultaneous Localization and Mapping.

Instead of focusing on state-of-the-art solutions to a particular sub-problem, emphasis of the book is on a progressive step-by-step development concepts through recurrent examples that capture the essence of a problem. The described solutions might not necessarily be the best, however they are easy to comprehend and widely used in the community. For example, odometry and line-fitting are used to explain forward kinematics and least-squares solutions, respectively, and later serve as motivating examples for error propagation and the Kalman filter in a localization context.

Notably, the book is explicitly robot-agnostic, reflecting the timeliness of fundamental concepts. Rather, a series of possible project-based curricula are described in an Appendix and available online, ranging from a maze-solving competition that can be realized with most camera-equipped differential-wheel robots to manipulation experiments with a robotic arm, all of which can be entirely conducted in simulation to teach most of the core concepts.

After multiple years of development and distribution mainly via Github, this new edition of the book has been co-authored by my colleagues in the Computer Science department, Bradley Hayes, Christoffer Heckman, and Alessandro Roncone, each having thaught multiple iterations of our ``Introduction to Robotics'' and ``Advanced Robotics'' courses as well as special topics courses that pertain to their sub-fields of robotics research. They are adding not only tremendous technical depth, but also years of experience on how certain subjects should be taught to remain engaging and exciting.

This book is released under a Creative Commons CC BY-NC-ND 4.0 International license, which allows anyone to copy and share its source code. However, neither the compiled version nor the code shall be used to create derivatives for commercial purposes. We have chosen this format as it seems to maintain the best trade-off between a freely available textbook resource that others may contribute to and maintaining a consistent curriculum that others can refer to. We are incredibly grateful to MIT Press and our editor Elizabeth Swayne to support this forward-looking model.

Writing this book would not have been possible without the excellent work of others before us, most notably ``Introduction to Robotics: Mechanics and Control'' by John Craig and ``Introduction to Autonomous Mobile Robots'' by Roland Siegwart, Illah Nourbakhsh and Davide Scaramuzza, and innumerable other books and websites from which I learned and borrowed examples and notation. We are also grateful to Brian Amberg, Aaron Becker, Bachir El-Kadir,  James Grime, Michael Sambol, Cyrill Stachniss, Subh83, Ethan Tiran-Thompson who made lecture video snippets and animations available online, and which are referenced throughout the book using QR codes.

I would like to acknowledge Mike Miles and Harel Biggie, graduate students in the authors' shared laboratory at the University of Colorado Boulder, for their careful reading and contributions. I would also like to thank Prof. Chaoqun Wang from Shandong University, China, for the Chinese translation.  Finally, I would also like to acknowledge Github users AlWiVo, beardicus, mguida22, aokeson, as1ndu, apnorton, JohnAllen, jmodares, countsoduku, choffmann, and chrstphrdlz for their pull requests and comments as well as Haluk Bayram. Your interest and motivation in this project has been one of our biggest rewards.

\begin{flushright}
Nikolaus Correll\\
Boulder, Colorado, \today
\end{flushright}

\input{chapters/introduction}

\part{Mechanisms}
\input{chapters/locomotion}
\input{chapters/kinematics}
%!TEX root = ../book.tex
\chapter{Forces}\label{ch:forces}
%alessandro

So far, we have only been concerned with how robots move and the \textsl{geometry of motion}.
However, moving a robot not only requires a kinematic model of the platform under consideration, but also an understanding of the (generalized) forces needed to actuate the robot's motors and those needed for the robot to interact with the environment.
While this aspect can be ignored in basic applications of mobile robots and simple manipulation, it becomes critical as soon as robots interact more closely with people or need to engage in more complex manipulation: in these scenarios, \textsl{safety} and \textsl{model accuracy} are of paramount importance.

%The (geometric) Jacobian represents a fundamental tool to characterize the motion and the interaction of a robot with its environment, as it is used to perform the following \cite{sciavicco2012modelling}):
%\begin{enumerate}
%\item inverse differential kinematics even for robots that do not have a closed-form solution (\cref{sec:invjac});
%\item singularity analysis (a kinematic singularity is a robot configuration in which the robot loses the ability to move to one or more directions);
%\item redundancy analysis (a kinematic task is redundant if the robot possesses more degrees of freedom than what are needed to perform the task, resulting in infinite inverse kinematics solutions to choose from);
%\item manipulability analysis (i.e. how easy or difficult is it for a robot to move in a certain direction).
%\end{enumerate}
In this Chapter, we will introduce the reader to these concepts through \textsl{statics}\index{Statics}, which introduces a third abstraction to the problem of analyzing how robots move in space and interact with their surroundings.
More specifically, in \cref{sec:kinematics:fwd,sec:kinematics:ik} we have investigated the \textsl{kinematic} problem and operated in the space of \textsl{positions}, that is, how to map joint angles with end effector poses.
In \cref{sec:kinematics:diff}, we introduced the \textsl{differential kinematics} problem and operated in the space of \textsl{velocities}, i.e. how to map joint velocities with end-effector velocity twists (remember: velocity is derivative of position, hence the name ``differential'').
In the following, we will operate in the space of \textsl{forces}; however, we will simplify the more general dynamical problem by looking at the robot in static equilibrium---otherwise known as a \textsl{static} configuration.
As we will see, a lot can be done by simply looking at the robot in an equilibrium configuration!
The fourth and last abstraction, which goes beyond the scope of this book, is called \textsl{dynamics} and operates in the space of forces from a non-static perspective; it involves the second derivative of position (i.e. the acceleration), and it can be thought as a generalization of the second law of Newton ($F=ma$).
%We will briefly introduce the reader to the topic in \cref{sec:forces:dynamics}.\td{remember to remove this sentence if we decide not to have the subchapter on dynamics}
%
The goals of this chapter are to:

\begin{itemize}
\item Introduce the concept of statics,
\item understand the so called ``kineto-statics duality'',
\item become familiar with the notion of ``manipulability''.
%\item briefly introduce the dynamics problem.
\end{itemize}

Most of the concepts below are typically considered in the context of manipulation, as mobile robots generally do not exchange forces with their environment.
Therefore, for simplicity we will hereinafter refer to robot manipulators equipped with revolute joints unless otherwise specified.

\begin{mdframed}
\noindent The analysis of motion of a robot can be thought as a layered system with multiple levels of abstraction of increasing complexity.
The more complex it becomes, the more comprehensive your analysis will be, and the more capability you will be able to squeeze out of the robot!
However, it is good practice to start with the simplest layer first (i.e. kinematics), and gradually progress toward a dynamic analysis only if needed.
\end{mdframed}

\section{Statics}\label{sec:forces:statics}

\textsl{Statics} deals with relating (generalized) forces at the robot's joints and generalized forces at the end-effector when the robot is in \textsl{static (or mechanical) equilibrium}\index{Mechanical Equilibrium}\index{Static Equilibrium}, i.e. the acceleration of the robot and all of its components is zero.
If such a condition is met, a robot with $n$ degrees of freedom and an end-effector characterized by $m$ degrees of freedom can be fully described by the following quantities:
\begin{itemize}
    \item an $\left( n \times 1 \right)$ vector of generalized forces $\tau$ at the joints;
    \item an $\left( m \times 1 \right)$ vector of generalized forces $F$ exerted \textsl{by the robot end-effector} on the environment---or, more generally, by any part of the robot that may be in direct physical contact with the environment;
    \item an $\left( m \times 1 \right)$ vector of forces exerted \textsl{by the environment} on the robot end-effector $F_e$---which, per the principle of action and reaction, are equal and opposite to $F$: $F_e=-F$.
\end{itemize}
In this case, \textsl{generalized force}\index{Generalized Force} means ``any force-equivalent quantity needed to describe the element''.
In the case of joints, it depends on the actuation: generalized forces at the joints are either forces for prismatic joints (as they impart a translational motion on the joint) or torques for revolute joints (as they impart a rotational motion on the joint); the size of this vector depends on the number of mechanical degrees-of-freedom $n$.
In the case of the end-effector, it depends on the number of DoFs in task space $m$; if we are operating with a $6-$DoF problem, the $m \times 1$ vector of generalized forces will be composed of a linear force component given by the forces on the three axes:
\begin{equation}\label{eq:wrench:force}
f=\left[\begin{array}{c}
f_x\\
f_y\\
f_z
\end{array}
\right],
\end{equation}
and an angular force component (or moment) $\mu$ around the three axes:
\begin{equation}\label{eq:wrench:moment}
\mu=\left[\begin{array}{c}
\mu_x\\
\mu_y\\
\mu_z
\end{array}
\right].
\end{equation}
We can now combine the above elements in a $6\times1$ vector as $F=[f \ \mu]^T$.
This vector of generalized forces is also called a \textsl{spatial force} or \textsl{wrench}\index{Wrench}\index{Spatial Force}.
We now want to compute the statics version of \cref{eq:kinematics:diff:fwd:short,eq:kinematics:diff:fwd}, and relate our $n \times 1$ vector of torques $\tau$ with the $6\times1$ wrench vector $F$.
To find this relationship, let's recall the definition of \textsl{power} from physics. Mechanical power $W$ is defined as force times velocity, which can be generalized as generalized forces times generalized velocities: $W=F^T \cdot \nu$.
Now, we know that the forces exchanged at the end-effector come from our source of actuation, i.e. our motors, whose generated power is defined by $W=\tau ^T \cdot \dot{q}$. We therefore have that:
\begin{equation}\label{eq:forces:statics:work}
W=\tau ^T \cdot \dot{q} = F^T  \cdot \nu
\end{equation}
We also know the relation between $\nu$ and $\dot{q}$ from \cref{eq:kinematics:diff:fwd:short}: $\nu =J(q) \cdot \dot{q}$. \cref{eq:forces:statics:work} then becomes:
\begin{equation}
\tau ^T \cdot \dot{q} = F^T  \cdot J(q) \cdot \dot{q} \ ,
\end{equation}
which, with minor rearrangements, turns into the following:
\begin{equation}\label{eq:forces:statics}
\tau = J(q) ^T \cdot F
\end{equation}
This is the final statics equation we were looking for!
It can be interpreted as the following: to counteract an external wrench $F_e = -F$ applied on the end effector by the environment in a static configuration $q$, the robot needs to apply torques $\tau$ at its joints as specified by \cref{eq:forces:statics}.
Interestingly, this equation clearly shows how statics acts as a middle ground between the ``geometry-only'' kinematics approach and the more general dynamics problem: even though we are dealing with forces and torques, their relationship is defined via a geometric relation---i.e. the same Jacobian used in \cref{sec:kinematics:diff:inv}. In this case, we are using its $n\times m$ transpose:

\begin{equation}\label{eq:forces:statics:long}
\tau = \left[\begin{array}{c}\tau_1\\\vdots\\\tau_n\end{array}\right] =
\left[\begin{array}{cccc}\frac{\partial{x}}{\partial{q_1}} & \frac{\partial{y}}{\partial{q_1}} & \ldots & \frac{\partial{\omega_z}}{\partial{q_1}}\\\frac{\partial{x}}{\partial{q_2}} & \frac{\partial{y}}{\partial{q_2}} & \ldots & \frac{\partial{\omega_z}}{\partial{q_2}}\\\vdots & \vdots & \vdots & \vdots\\\frac{\partial{x}}{\partial{q_n}} & \frac{\partial{y}}{\partial{q_n}} & \ldots & \frac{\partial{\omega_z}}{\partial{q_n}}\end{array}\right]\left[\begin{array}{c}f_x\\f_y\\f_z\\\mu_x\\\mu_y\\\mu_z\end{array}\right] = J(q) ^T \cdot F
\end{equation}

\cref{eq:forces:statics} is useful on a variety of different problems. The most typical application is \textsl{force control}\index{Force Control}, i.e. the robot's motors are actuated so as to apply a specified wrench on the environment. For example, one may want to use a robot for a polishing task in which it needs exert a vertical force of $5N$ on a table. In this case, the desired wrench (assuming our $z-$ axis in Cartesian space is the vertical one and it is pointing upwards) would be:

\begin{equation}
F=\left[\begin{array}{c} 0\\ 0\\ -5N\\ 0\\ 0\\ 0\\ \end{array} \right].
\end{equation}

% ---either torques for revolute joints or forces for prismatic joints---

% joint torques $\tau$


% In this Chapter, we will investigate the role of the Jacobian in relating joint torques $\tau$ with forces and moments $[f \ m]^T$ applied at the end-effector--and we are going to do all of this in equilibrium (or \textsl{static}) configurations.

\section{Kineto-Statics Duality}\label{sec:forces:kinetostatics}

The analogy between \cref{eq:kinematics:diff:fwd,eq:forces:statics} by means of the Jacobian makes it interesting to analyze \cref{eq:forces:statics} similarly to what we did in \cref{sec:kinematics:diff,sec:kinematics:diff:underover}.
This analogy is defined as \textsl{kineto-statics duality}\index{Kineto-Statics Duality}, and helps the novice roboticist to more intuitively correlate these two levels of abstraction.
More specifically, singular configurations are as relevant to the statics problem as they are to the differential kinematics one, but they have different physical interpretation.
In a singular configuration, both the Jacobian and its transpose lose rank---as transposing a matrix does not affect its rank. However, while loss of full rank affects the \textsl{inverse} kinematics problem (i.e. its solution ``explodes'' and joint speeds go to infinity), in this case it is the \textsl{direct} statics mapping that is affected by it: in a singular configuration, forces exerted by the robot on the environment go to infinity.
This is an additional (and arguably more compelling) reason to avoid singularities at all costs: the robot would move very fast \textsl{and} exert strong forces on anything on its path.

\section{Manipulability}\label{sec:forces:manipulability}

The duality property that exists between differential kinematics (\cref{sec:kinematics:diff}) and statics (\cref{sec:forces:statics}) allows us to further inspect manipulator performance for a given joint configuration.

\subsection{Manipulability Ellipsoid in Velocity space}

As a first step, we may inspect the capacity of the manipulator to arbitrarily change its end-effector's position and orientation from the current configuration.
More specifically, we may ask the following question: what effect does a small increment in joint positions (i.e.\ a small joint velocity) have on the end-effector pose?
Let's consider the set of joint velocities of unit norm defined by the following equation:
\begin{equation}
\dot{q}^T\cdot\dot{q} = 1 \quad .  \label{eq:forces:manipulability:velocitysphere}
\end{equation}
This equation represents a multi-dimensional ``sphere'' in joint space $\mathbb{R}^n$.
We know from \cref{sec:kinematics:diff} that this corresponds to a similarly multidimensional shape in operational space $\mathbb{R}^m$, and we know that this correspondence is mediated by \cref{eq:kinematics:diff:fwd:short} and its inverse. In the generic case of a redundant manipulator, \cref{eq:forces:manipulability:velocitysphere} becomes:
\begin{equation}
\nu^T J(q)^{+T} \cdot J(q)^+ \nu = 1 \quad ,
\end{equation}
which, combined with \cref{eq:kinematics:diff:pseudoinverse}, becomes:
\begin{equation}
\nu^T \left[ J(q) \cdot J(q)^T \right]^{-1} \nu = 1 \quad ,
\label{eq:forces:manipulability:velocitymanipulability}
\end{equation}
which corresponds to a multidimensional ellipsoid in operatinal space $m$---otherwise known as \textsl{velocity manipulability ellipsoid}\index{Velocity Manipulability Ellipsoid}. This ellipsoid provides the following physical interpretations:

\begin{itemize}
\item Along the direction of its major axis, the robot can move at large velocities;
\item Along the direction of its minor axis, the robot can move at small velocities;
\item The volume of the ellipsoid is called \textsl{manipulability measure} and is defined as $w(q)=\sqrt{det\left[ J(q)J(q)^T \right]}$. It is always positive and it reaches a maximum when the ellipsoid is close to a sphere and the robot can move isotropically in any direction.
\item In a singularity, the ellipsoid ``loses a dimension'' and one of its axis becomes 0. Concurrently, the manipulability measure $w(q)=0$---which is why $w(q)$ is used to understand how far a robot is from a singular configuration.
\end{itemize}

The properties above can easily be verified in \cref{fig:manipulability}, top. The closer the robot is to a singular configuration (for example, the arm fully stretched), the more the $2$-dimensional ellipsoid converges to a vertical line. At the singularity itself, the minor axis has zero length, signifying that the robot can only move on a vertical direction and not right or left.

\begin{figure}[!t]
    \centering
    \def\svgwidth{0.8\textwidth}
    \import{./figs/}{manipulability.pdf_tex}
    \caption{Velocity (top) and force (bottom) manipulability ellipsoids for a 2 DoF planar arm ($n=m=2$). In this simple $2\times2$ case, the ellipsoids collapse to simple ellipses (whose major and minor axes are drawn in a dotted line).}\label{fig:manipulability}
\end{figure}

\subsection{Manipulability Ellipsoid in Force space}

By virtue of the kineto-statics duality, similar considerations can be done in force space. In this case, we may want to consider a sphere in the space of joint torques:
\begin{equation}
\tau^T\cdot\tau = 1 \quad ,  \label{eq:forces:manipulability:torquesphere}
\end{equation}
which, thanks to \cref{eq:forces:statics}, is mapped into a \textsl{force manipulability ellipsoid}\index{Force Manipulability Ellipsoid}:
\begin{equation}
F^T \left[ J(q) \cdot J(q)^T \right] F = 1 \quad ,
\label{eq:forces:manipulability:forcemanipulability}
\end{equation}
This ellipsoid characterizes the forces at the end-effector that can be exerted by the robot on the environment in the given joint configuration $q$. It behaves similarly to the velocity manipulability ellipsoid, with one important difference: while the principal axes of both ellipsoids are in the same orientation, their magnitude is in inverse proportions.
As depicted in \cref{fig:manipulability} (bottom), the major axis in force space becomes the minor axis in velocity space and vice versa.
Therefore, a direction of high velocity manipulability corresponds to a direction of low force manipulability.

\subsection{Manipulability Considerations}

The velocity and force manipulability ellipsoids are useful for a variety of tasks, from identifying a suitable joint configuration to perform a specific task, to understand what it is possible for the robot to do in a specific configuration. Remember that, for a kinematically redundant manipulator (see \cref{sec:kinematics:diff:underover}), it is possible to be in the same task space configuration (e.g. end-effector pose) with multiple joint postures: therefore, a manipulability analysis may allow the robot designer to choose the configuration that better conforms to additional specifications (e.g. lower exerted force, lower energy consumption, better legibility of robot motion by humans, and more).

% \section{Force interactions and compliance}

% \section{Dynamics}\label{sec:forces:dynamics}


\section*{Take-home lessons}
\begin{enumerate}
\item Looking at forces in mechanical equilibrium, that is when end-effector forces and joint torques cancel each other, allows us to extend control of the robot from poses and velocities to the force domain.
\item The torques required by a robotic arm are related to end-effector forces using the same Jacobian that also defines the robot’s differential kinematics---a concept known as the Kineto-Statics Duality.
\item Although a robotic arm might reach a desired pose using multiple different configurations, some configurations are better suited than others; a manipulability analysis helps in characterizing this problem.
\end{enumerate}

\section*{Exercises}\small

\begin{enumerate}
\item Think about the four layers of abstraction we have just investigated (kinematics, differential kinematics, statics, dynamics).
\begin{enumerate}
\item Can you think of an application for which you would need a dynamic analysis? (Hint: this is generally something really hard)
\item What can be done by just looking at the static problem instead? (Hint: you are still considering an exchange of forces here)
\item What can you do with a robot from a purely kinematic perspective? (Hint: this is typically easy)
\end{enumerate}
\item Why are singular configuration dangerous for the robot and its surroundings? Think about the relationship between forces and velocities.
\item How can you ensure the robot ``stays away'' from singularities?
\item Program an application that displays the manipulability ellipsoids in force and velocity for a two-link planar arm (similar to \cref{fig:manipulability})---feel free to integrate this program with the kinematics exercise in \cref{chap:kinematics}. How does the manipulability ellipsoid relate to positional increments of the end-effector? What happens in a singularity? (Hint: the easiest singularity to find for robot manipulators is the ``stretched out'' configuration)
\item Use a robot simulator of your choice to access a robot manipulator with at least three DoFs in joint space that moves in 3D. How does the manipulability ellipsoid change in this case? (Hint: it is not an ellipse any more)
\item A manipulability analysis is purely geometrical and depends on joint configuration of a given kinematics. Therefore, it is possible to use this analysis to characterize other (non-traditional) robot ``arms'' as well. Think about a biomechanical analysis of the human arm: in which configurations you have maximum manipulability? Which configurations correspond to high exertion (i.e. high ``torques'') resulting in small exerted forces on the environment?
\end{enumerate}\normalsize

\input{chapters/grasping}

\part{Sensing and actuation}
\input{chapters/actuators}
%!TEX root = ../book.tex
\chapter{Sensors}\label{chap:sensors}

Robots are systems that sense, actuate, and compute. So far, we have studied the basic physical principles of motion, i.e., locomotion and manipulation. We now need to understand the basic principles of robotic sensors that provide the necessary data for a robot to make decisions and control itself.
%
The goals of this chapter are to:
\begin{itemize}
\item provide an overview of sensors commonly used on robotic systems,
\item outline the physical principles behind the functioning of sensors, and
\item clarify the mechanisms responsible for uncertainty in sensor-based reasoning.
\end{itemize}

% \section{Robotic Sensors}
Historically, the development of robotic sensors is driven by industries other than robotics. These include transportation (automotive, naval, airplanes), safety devices for industry, servos for remote-controlled toys, and more recently cellphones, virtual reality, and gaming consoles. These industries are mostly responsible for making ``exotic'' sensors available at low cost by identifying mass-market applications.
For example, accelerometers and gyroscopes are now widely used in smartphones and cost less than a dollar; the XBox gaming console made 3D depth sensing (through the Kinect) accessible for a greatly lower cost than before; and sensors in modern passenger vehicles provide an array of capability without appreciably increasing the cost of the vehicle itself.

\begin{mdframed}
Think about the sensors that you are interacting with daily. What sensors do you have in your phone, in your kitchen, or in your car?
\end{mdframed}

As we will see later on, sensors are hard to classify by their application domain and target use case. In fact, most problems benefit from every possible source of information they can obtain. For example, localization can be achieved by measuring how many degrees a wheel has turned with a sensor known as an ``encoder'' (\ref{sec:sensors:encoders}), a sensor that can be implemented relying on a wide variety of physical principles. However, estimation becomes more precise with the addition of accelerometers (\cref{sec:sensors:inertia}) or even vision sensors (\cref{chap:vision}). All of these approaches differ drastically in their precision---a term that will be more formally introduced below---and the kind of data they provide, but none of them is able to completely solve the localization problem on its own.


\begin{mdframed}
Think about the kind of data that you can obtain from an encoder, an accelerometer, or a vision sensor on a non-holonomic robot. What are the fundamental differences? What physical principles do they leverage?
\end{mdframed}

Although an encoder is able to measure position, it is used in this function only on robotic arms. If the robot is non-holonomic, closed paths in its configuration space (i.e., robot motions that return the encoder values to their initial position), do not necessarily drive the robot back to its starting point (as exemplified in \cref{fig:holonomy}).
In those robots, encoders are therefore mainly utilized to measure speed. An accelerometer instead, by definition, measures the derivative of speed. Vision, finally, allows for the calculation of the absolute position (or the integral of speed) if the environment is equipped with unique features. An additional fundamental difference between those three sensors is the amount and kind of data they provide. An accelerometer samples real-valued quantities that are digitized with some precision. An odometer instead delivers discrete values that correspond to encoder increments. Finally, a vision sensor delivers an array of digitized real-valued quantities (namely colors). Although the information content of this sensor exceeds that of the other sensors by far, cherry-picking the information that is the most useful to complete a task remains a hard, and generally unsolved, problem.

\section{Terminology}\label{sec:sensors:terminology}

When dealing with sensors, it is important to provide precise definitions of terms such as ``speed'' and ``resolution'', as well as additional taxonomy that is specific to robotics.
%
Roboticists differentiate between \textsl{active} and \textsl{passive} sensors. Active sensors \index{Active sensor} emit energy of some sort and measure the reaction of the environment. Passive sensors \index{Passive sensor} instead measure energy from the environment. For example, most distance sensors (not including stereo vision) are active sensors, because they sense the reflection of a signal they emit; conversely, an accelerometer, a compass, or a push-button are passive sensors. Frequently the addition of an active element to a passive sensor can increase the signal-to-noise ratio of the passive sensor, so these distinctions may be blurred in some cases.

Another important term to characterize sensors is its \textsl{range}\index{Range (sensor)}, i.e. the \textsl{difference} between the upper and the lower limit of the quantity a sensor can measure.
This differs from its \textsl{dynamic range}\index{Dynamic Range (sensor)}, which is the \textsl{ratio} between the highest and lowest value a sensor can measure. It is usually expressed on a logarithmic scale (to the basis 10), also known as ``decibel''\index{Decibel}. The minimal distance between two values a sensor can measure is known as its \index{Resolution (sensor)} \textsl{resolution}. The resolution of a sensor is primarily limited by the physical principle it leverages (e.g., a light detector can only count multiples of a quant), however it is usually limited by the analog-digital conversion process.
The resolution of a sensor should not be confused with its accuracy or its precision (which are two different concepts). For example, even though an infrared distance sensor might produce $4096$ different values to encode distances from $0$ to $10cm$ (which suggests a resolution of around $24$ micrometers), its precision is much lower than its resolution (usually in the order of millimeters) due to noise in the acquisition process.

A sensor's \textsl{accuracy}\index{Accuracy (sensor)} is the difference between its (average) output $m$ and the true value $v$ to be measured:
\begin{equation}
accuracy=1-\frac{|m-v|}{v}
\end{equation}
This measure provides a quantity that approaches $1$ for very accurate values and $0$ if the measurements group far away from the actual value. In practice, however, this measure is rarely used and accuracy is provided with absolute values or a percentage at which a value might exceed the true measurement.

A sensor's \textsl{precision}\index{Precision (sensor)} instead is given by the ratio of range and statistical variance of the signal.
As detailed in \cref{fig:precision}, precision is therefore a measure of \textsl{repeatability} of a signal, whereas accuracy describes a \textsl{systematic error} that is introduced by the sensor's physics.
\begin{figure}
	\centering
	% \includegraphics[width=0.9\textwidth]{figs/precisionvsaccuracy.png}
	\def\svgwidth{0.9\textwidth}
    \import{./figs/}{precisionvsaccuracy.pdf_tex}
	\caption{The cross corresponds to the true value of the signal. From left to right: neither precise nor accurate, precise but not accurate, accurate but not precise, accurate and precise.
	\label{fig:precision}}
\end{figure}
%
For example, a GPS sensor is usually precise within a few meters, but only accurate to tens of meters. This becomes most obvious when satellite configurations change, resulting in the precise region jumping by a couple of meters. In practice, this can be avoided by fusing this data with other sensors, e.g.\ from an IMU.

The speed at which a sensor can provide measurements is known as its \index{Bandwidth (sensor)} \textsl{bandwidth}. For example, if a sensor has a bandwidth of 10 Hz, it will provide a signal ten times a second. This is important to know, as querying the sensor more often is a waste of computational time and potentially misleading.


\subsection{Proprioception vs. Exteroception}
\label{sec:sensors:proprioception}

Another important taxonomy is the difference between proprioception and exteroception.
\textsl{Proprioception}\index{Proprioception} refers to the perception of the internal state of a robot.
Proprioception includes estimation of the robot's joint angles, its speeds, as well as internal torques and forces.
%
% Finally, there are other means of proprioception, such as simple sensors that can detect when a robot gets picked up, e.g.
%
%\section{Exteroception of the physical interaction with the environment}
\label{sec:sensors:interaction}
%
Conversely, \textsl{exteroception}\index{Exteroception} refers to sensing anything outside of the physical embodiment the robot. Exteroception is important because it is crucial for the robot to correctly perceive the state of the world, estimate the uncertainties related to it, and properly act based upon these uncertainties.
Importantly, while the majority of sensor development focuses on \textsl{distal} sensors capable of measuring quantities in the far space (e.g. cameras, see \cref{chap:vision}, or sound-based sensors, see \cref{sec:sensors:sound}), in recent years more attention has been given to \textsl{proximal} sensors, that are concerned with measuring the environment that is immediately surrounding the body or even directly on the robot body.
Applications of this technology are varied, from tasks that require measuring and controlling the interaction of the end-effector with the environment (e.g. sanding a table with a fixed vertical force), to manipulating in clutter---where contact with obstacles is inevitable.

\begin{mdframed}
In robotics, it often helps to make comparisons with human performance.
How many daily tasks \textsl{do not} require physical interaction with the environment?
If they do, would you be able to successfully complete them without contact, and how would your performance decrease if you were to ``avoid collisions at all costs''?
\end{mdframed}

%\section{Proprioception of the internal state of the robot}\label{sec:sensors:proprioception}

\section{Sensors that measure the robot's joint configuration}\label{sec:sensors:encoders}

The most important proprioceptive sensor is the \textsl{encoder}\index{Encoder}. Encoders can be used for sensing joint position and speed, as well as force---if used in conjunction with a spring. Encoders can be divided in incremental (relative, used primarily in mobile robotics) and absolute encoders (mainly used in robot manipulators).
%The latter are mostly used in industrial applications, but are not common in mobile robotics.
In general, they rely on either a magnetic or optical beacon turning together with the motor and being sensed by an appropriate sensor that counts every pass-through. The most common encoder in robotics is the \textsl{quadrature encoder}\index{Quadrature encoder}, which is an optical encoder. It relies on a pattern rotating with the motor and an optical sensor that can register black/white transitions, as shown in \cref{fig:encoders}.

\begin{figure}
	\centering
		\includegraphics[width=0.3\textwidth]{figs/encoderdisk.png}
		\fontsize{7pt}{9pt}\selectfont		
		\def\svgwidth{0.3\textwidth}
    \import{./figs/}{quadraturencoder.pdf_tex}	
%		\includegraphics[width=0.3\textwidth]{figs/quadraturencoder.png}
		\includegraphics[width=0.3\textwidth]{figs/absoluteencoder.png}
	\caption{From left to right: encoder pattern used in a quadrature encoder, resulting sensor signal (forward motion), absolute encoder pattern (gray coding).}
	\label{fig:encoders}
\end{figure}

While a single sensor would be sufficient to detect rotational position and speed, it does not allow to determine the direction of motion. Quadrature encoders therefore have two sensors, A and B, that register an interleaving pattern with distance of a quarter phase. If A leads B, the disk is rotating in a clockwise direction. If B leads A, then the disk is rotating in a counter-clockwise direction. It is also possible to create absolute encoders---an example of which is shown in \cref{fig:encoders}, right. This pattern is a 3-bit pattern that encodes 8 different segments on a disc. Notice that the pattern is arranged in such a way that there is only one bit changing from one segment to the other. This is known as ``Gray code''\index{Gray code}.
% The function of linear encoders is analogous, both for incremental and absolute encoders.

\section{Sensors that measure ego-motion}\label{sec:sensors:inertia}

Measuring the robot's joint configuration is limited to static observations. It does not allow the robot to detect whether it is currently moving or even accelerating (such as falling), which is particularly important for robots that are only dynamically stable such as walking humanoids or quadrotors. Motion can be estimated by relying on the principle of \emph{inertia}. A moving mass does not lose its kinetic energy---if there is no friction. Likewise, a resting mass will resist acceleration. Both effects are due to inertia \index{Inertia} and can be exploited to measure acceleration and speed.

\subsection{Accelerometers}

An accelerometer\index{Accelerometer} can be thought of as a mass on a dampened spring. Considering a vertical spring with a mass attached to it, we can measure the acting force $F=kx$ (Hooke's law\index{Hooke's law}) by measuring the displacement $x$ that the mass has exerted on the spring.
Using the relationship $F=ma$, we can now calculate the acceleration $a$ on the mass $m$. On earth, this acceleration is roughly $9.81\frac{m}{s^2}$.
In practice, these spring/mass systems are realized using microelectromechanical devices (MEMS), such as a cantilevered beam whose displacement can be measured using a capacitive sensor. Accelerometers measure up to three axes of translational accelerations. Inferring an absolute position from it requires a double integration, which introduces significant noise in the estimation and makes position estimates using accelerometers infeasible in practice.
However, as gravity provides a constant acceleration vector, accelerometers are very good at estimating the pose of an object with respect to gravity (i.e.\ roll and pitch).

\subsection{Gyroscopes}\label{sec:gyroscopes}

A gyroscope is an electro-mechanical device that can measure rotational speed, and in some configurations orientation. It is complementary to the accelerometer that measures translational acceleration. Classically, a gyroscope consists of a rotating disc that can freely rotate in a system of pivots and gimbals. When moving the system, the inertial momentum maintains the original orientation of the disc, allowing to measure the orientation of the system relative to where the system was originally. While disc-based gyroscopes are still used, for example to stabilize the cannon of a tank during motion, the mechanism remains difficult to minimize.

A variation of the gyroscope is the rate gyro, which measures rotational/angular velocities. A rate gyro \index{Rate gyro}\index{Gyroscope} can intuitively be illustrated by considering its \textsl{optical} implementation.
In an optical gyro, a laser beam is split in two and sent around a circular path in two opposite directions. If the device is rotated against the direction of one of these laser beams, one laser will have to travel slightly longer than the other, leading to a measurable phase shift at the receiver.
This phase shift is proportional to the \textsl{rotational speed} of the setup. As light with the same frequency and phase will add to each other and lights with the same frequency but opposite phases will cancel each other, light at the detector will be darker for high rotational velocities.
Importantly, small-scale optical rate gyros are not practical, but MEMS rate gyros are widely available and use a different technology, as they rely on a mass suspended by springs. The mass is actively vibrating, making it subject to Coriolis forces when the sensor is rotated. Coriolis forces can be best understood by moving orthogonally to the direction of rotation on a vinyl disk player. In order to move in a straight line, you will not only need to move forwards, but also sideways. The necessary acceleration to change the speed of this sideways motion is counteracting the Coriolis force, which is both proportional to the lateral speed (the vibration of the mass in a MEMS sensor) and the rotational velocity, which the device wishes to measure. Note that the MEMS gyro would only be able to measure acceleration if it were not vibrating.

Rate gyroscopes can measure the rotational speed around three axes, which can be integrated to obtain absolute orientation. As an accelerometer measures along three axes of translation, the combination of both sensors can provide information on motion in all six degrees of freedom. Together with a magnetometer (compass), which provides absolute orientation, this combination is also known as \textsl{Inertial Measurement Unit}\index{Inertial Measurement Unit} (or IMU\index{IMU}). Note that this combination of sensors is particularly powerful, as an accelerometer and gyroscope can provide complementary information on roll and pitch, while a magnetometer and gyroscope can provide complementary information on yaw. This innovation has powered attitude and heading reference systems (AHRS) through sensor fusion, a technique that is explored in Section \cref{sec:sensors:globalpose}.

\section{Measuring force}
The measurement of physical interaction forces is of paramount importance for robotics.
It enables a variety of capabilities that humans take from granted, from gently picking a strawberry to safely engaging in touch-based interactions with humans.

When combining a motor and an encoder with a spring, a mechanism known as  a \textsl{Series Elastic Actuator} \cite{pratt1995series} \index{Series Elastic Actuator}, rotary and linear encoders can be used as simple force or torque sensors using Hooke's law ($F=kx$, where $k$ is a spring constant) and $x$ the displacement in the spring due to extension or compression.
This can be used when operating a robot under a static (\cref{sec:forces:statics}) or dynamic level of abstraction.
% Whereas the series elastic actuator is the most illustrative examples, most load cells operate on the premise of measuring displacements within materials of known properties. Here, measuring changes in resistance or capacitance might be easier choices.
Another method to estimate the actual force or torque acting on a joint is to measure the current consumed at each joint. Knowing a mechanism's pose allows to calculate the resulting forces and torques across the mechanism as well as the currents required for empty loading conditions. Derivations of these then correspond to additional forces that can hence be calculated.


\begin{figure}
	\centering
	\def\svgwidth{0.9\textwidth}
    \import{./figs/}{ftsensor.pdf_tex}
	\caption{A force/torque sensor translates force and torque between two links of a robot via three metal rods that connect an inner hub to an outer ring. Here, one link of the robot connects to the outer ring, the other to the inner hub.  Each metal rod is equipped with a strain gauge on each side, resulting in 12 sensors in total. 
	\label{fig:ftsensor}}
\end{figure}


The most accurate and most widespread device to date is the \textsl{Force/Torque (or F/T) sensor}\index{Force/Torque sensor}. It is a mechanical device that is capable of detecting one or more components of a six-dimensional wrench applied to it (i.e. a 3D force and a 3D torque, see \cref{sec:forces:statics}).
Most commercially available F/T sensors use  \textsl{strain gauges}. Simply put, a strain gauge is a metal (i.e. conductive) foil that changes its shape when a wrench is applied to it, and while doing so changes its electrical resistance. In a typical configuration (\cref{fig:ftsensor}), a F/T sensor consists of an inner hub of solid metal that is suspended in an outer ring via three symmetrical, rectangular solid metal rods. Each metal rod is equipped with a strain gauge on each side (four per rod). Typically, sensors are operated in pairs, one mounted orthogonal to the other, resulting in a total of six sensor signals, from which we can compute forces and torques in three dimensions. Such F/T sensors are available as stand-alone components that can be mounted between an end-effector and a robot arm or are integrated within the robot's joints. 

While accurate and precise, F/T sensors are plagued by a number of limitations: 1) high costs due to the high precision that is required during manufacturing; 2) size (a standard F/T sensor is usually the size of a human wrist); 3) low signal-to-noise ratio; 4) low bandwidth/responsivity; and 5) a single data point that is sparse in space and time. This last point becomes particularly clear when considering a robotic arm having multiple points of contact with an object. Here, a single sensor that measures forces and torques at the joint provides only very little information. 

\subsection{Measuring pressure or touch}
In order to partially mitigate these limitations, roboticists have worked on a complementary capability, that is measuring the pressure applied on the robot's surface.

The human sense of touch is the oldest, the largest, and the most important of our senses.
To humans, contact and physical interaction are a resource rather than an impediment, and we are surprisingly proficient at leveraging touch in a variety of situations.
Therefore, it is natural for roboticists to equip robots with similar capabilities in order to achieve performance levels comparable to humans'.

A pressure sensor is a device that is capable of detecting either a contact/collision as a binary data (in which case is generally referred to as touch sensor), or a gradient of pressure applied to it.
In general, the vertical pressure applied to the sensor is proportional to the 1D vertical force that is applied to the direction normal to the sensor, and this makes a pressure sensor a good substitute to F/T sensors in specific applications (e.g. grasping).
Additionally, pressure sensors are mostly based on measuring pressure through a change in \textsl{capacitance} rather than resistance (no different from the functioning of a touchscreen on a modern smartphone): when pressure is applied to a capacitor-like device (i.e. two conductive plates separated by an insulating material), the distance between the two plates reduces and this causes a change in capacitance which can be readily measured.

As introduced in the context of series-elastic actuators, distance and force sensing are tightly related via Hooke's law. There exists a large variety of touch and force sensors that rely on light-based distance measurement (see the following section) in conjunction with a flexible material with known spring constant, such as using distance sensors to measure the deformation of an elastic dome from the inside \cite{youssefian2013contact} or measuring distance to objects through transparent rubber before touch and contact force after touch \cite{patel2018integrated}.

If compared to F/T sensors, pressure sensors provide limited amount of information ($1$-dimensional vs $6$-dimensional), but they allow for: 1) high responsiveness; 2) high density of sensing (up to tens of sensors per $cm^2$); 3) low cost; 4) ease of miniaturization.

Human touch is not limited to pressure alone, but also high-frequency information such as vibrations. These are important when discerning different surfaces. Robots can replicate this capability by measuring vibrations in the order of hundreds of Hertz by integrating accelerometers or microphones into a soft transducer and classifying spectral information \cite{hughes2015texture}.

In an extreme, it might be desirable to equip robots with an \emph{artificial skin}\index{Artificial skin} that combines different sensing modalities for pressure, texture, temperature or light, possibly also including cameras or actuators to change their appearance. While there exist a variety of commercial solutions, including pressurized double-layer skin that measure pressure differentials at select locations to detect contact, as well as capacitive solutions, robotic skins have not found wide-spread applications as of yet.  

%\subsection{Artificial skins for robotics}
%To date, there are no commercially available whole-body artificial skins for robots; however, a number of efforts (from both established robot manufacturers and startups) are showing promising prototypes. #THERE ARE TWO COMPANIES BLUE DANUBE (AIR PRESSURE) AND A FRENCH ONE (CAPACITIVE), 

%Interestingly, researchers are taking full advantage of the recent developments in a variety of different technologies, and are creating dense arrays of artificial sensors that combine one or more of the following sensing modalities: interaction forces (\cref{sec:sensors:interaction}), sound (\cref{sec:sensors:sound}), light (\cref{sec:sensors:light}), cameras (\cref{chap:vision}), inertia-based sensors (\cref{sec:sensors:inertia}), and more.

\section{Sensors to measure distance}
We have seen that there is a fluent transition from proprioceptive to exteroceptive sensors as measuring the robot's internal state is tightly related to its environment as soon as contact is made. In order to explore its environment from afar, measuring distance to individual objects has shown to be critical for the robot to navigate and identify obstacles and objects of interest. 

%\subsection{Light-based distance sensors}
%\label{sec:sensors:light}

The small form factor and low price of light-sensitive semi-conductors have led to a proliferation of light-based sensors relying on a multitude of physical effects. These include reflection, phase shift, and time of flight. Other physical principles that are commonly used in distance sensors are radio (more commonly known as ``Radar'') and sound.   

\subsection{Reflection}
Reflection is one of the easiest and most immediate principles to take advantage of: the closer an object is, the more it reflects light, radio or sound directed at it. This allows to easily measure distance to objects that reflect the signal well and that are not too far away. In order to make these sensors as independent from an object's color as possible (but unfortunately not totally independent), infrared is most commonly chosen wavelength when using light. In contrast, sound will not be effected by a surface's color, but by its surface properties and absorption characteristics. 

A reflection-based distance sensor is made from two components: an emitter (that emits a signal, for example infrared light) and a receiver (tasked with measuring the strength of the reflected signal). A typical response for an infrared distance sensor is shown in Figure~\ref{fig:epuckir}. The values obtained at an analog-digital converter correspond to the voltage at the infrared receiver and are saturated for low distances (flat line), and quadratically decrease afterwards.


\begin{figure}
	\centering
		\includegraphics[width=0.8\textwidth]{figs/epuckirsensor.png}
	\caption{Real-world response of an infrared distance sensor as a function of distance. Units are left dimensionless intentionally.}
	\label{fig:epuckir}
\end{figure}


\subsection{Phase shift}\label{sec:phaseshiftsensors}

As shown in Figure~\ref{fig:epuckir}, reflection can only be precise if distances are short. Instead of measuring the strength (amplitude) of the reflected signal, laser distance sensors measure the phase difference of the reflected wave. In order to do this, the emitted light is modulated with a wave-length that exceeds the maximum distance the scanner can measure.
If you were to use visible light and to do so at much slower speeds, you would see a light that keeps getting brighter, then getting darker, briefly turns off and then starts getting brighter again.

Thus, if you were to plot the amplitude of the emitted signal over time (i.e., its brightness), you would see a wave that crosses zero when the light is dark.
As light travels, this wave propagates through space with a constant distance (the wavelength) between its zero crossings. When it gets reflected, the same wave travels back (or at least parts of it that get scattered right back). For example, modern laser scanners\index{Laser range finder} emit signals with a frequency of 5 MHz (turning off 5 million times in one second). Together with the speed of light of approximately 300,000$km/s$, this leads to a wavelength of $60m$ and makes such a laser scanner useful up to $30m$.

When the laser is now at a distance that corresponds exactly to one half the wave-length, the reflected signal it measures will be dark at the exact same time its emitted wave goes through a zero-crossing. Going closer to the obstacle results in an offset that can be measured. As the emitter knows the shape of the wave it emitted, it can calculate the phase difference between emitted and received signal. Knowing the wave-length it can now calculate the distance. As this process is mostly independent from ambient light, the estimates can be very precise.
% This is in contrast to a sensor that uses signal strength. As the signal strength decays at least quadratically, small errors, e.g.\ due to fluctuations in the power supply that drives the emitting light, noise in the analog-digital converter, or simply differences in the reflecting surface have drastic impact on the accuracy and precision (see below for a more formal definition of this term).

As the laser distance measurement process is fast, such lasers can be combined with rotating mirrors to sweep larger areas, known as \textsl{Laser Range Scanners}\index{Laser Range Scanners} or \textsl{Lidars}\index{Lidar}. Such systems have been combined into packages consisting of up to $64$ scanning lasers and are nowadays vastly used in the autonomous driving space as they are capable of providing voluminous depth data of the environment around a car while driving.
It is also possible to modulate projected images with a phase-changing signal, which is the operational principle of early ``time-of-flight'' cameras, which however is not an accurate description of their operation.

\subsection{Time-of-flight}

The most precise distance measurements light can provide is by measuring its time of flight. This can be done by counting the time a signal from an emitter becomes visible in a receiver. As light travels very fast ($3\cdot10^8m/s$), this requires high-speed electronics that can measure time periods smaller than nanoseconds in order to achieve centimeter accuracy.
In practice, this is done by combining the receiver with a very fast electronic shutter that operates at the same frequency of the emitted light. As this timing is known, one can infer the time light has traveled by measuring the quantity of photons that made it back from the reflective surface within one shutter period.
As an example, light travels $15m$ in $50ns$. Therefore, it will take a pulse of $50ns$ to return from an object at a distance of $7.5m$. If the emitter transmits a pulse of $50ns$ length and then closes the receiver with a shutter, the receiver will receive more photons the closer the object is, but no photons if the object is farther than $7.5m$. Given a fast enough and precise circuit that acts as a shutter, it is sufficient to measure the actual amount of light that returns from the emitter.

\subsubsection{Ultrasound distance sensors}
\label{sec:sensors:sound}

Measuring the time of flight is considerably simpler when using sound waves to measure distance (sound travels at around $344m/s$ in air). An ultrasound distance sensor operates by emitting an ultrasound pulse and measures its reflection. Unlike a light-based sensor that measures the amplitude of the reflected signal, a sound-based sensor measures the time it took for the pulse to travel back and forth.
This is possible because sound travels at much lower speed ($3\cdot10^2m/s$) than light ($3\cdot10^8m/s$). The fact that the sensor actually has to wait for the signal to return leads to a trade-off between range and bandwidth (see \cref{sec:sensors:terminology}: allowing a longer range requires waiting longer for the signal to come back, which in turn limits how often the sensor can provide a measurement.
% (Look these definitions up above before you read on.)\td{fix--up above where?}
Although ultrasound distance sensors have become progressively less common in robotics, they have an advantage over light-based sensors: instead of sending out a ray, the ultrasound pulse results in a cone with an opening angle of $20$ to $40$ degrees. Because of this, such sensors are able to detect small obstacles without the requirement of directly hitting them with a ray. This property makes them the sensor of choice in specific applications, such as the automated parking assist technologies in modern cars.



\section{Sensors to sense global pose}
\label{sec:sensors:globalpose}
So far, we have discussed sensors that allow the robot to measure the position of its own joints,  its rotational velocity, its translational acceleration, forces from interaction with the environment, and distance to objects relative to its own pose. In order to reliably navigate in the environment, robots also need some notion of a world coordinate frame.  

Localizing an object by triangulation goes back to ancient civilizations, where sailors oriented themselves using the stars. As stars are only visible during unclouded nights, seafarers have invented systems of artificial beacons emitting light, sound, and eventually radio waves. The most sophisticated of such systems is the Global Positioning System (GPS). GPS consists of a number of satellites in orbit that are equipped with knowledge about their precise location and have synchronized clocks. These satellites broadcast a radio signal that travels at the speed of light and is coded with its time of emission. GPS receivers can therefore calculate the distance to each satellite by comparing time of emission and time of arrival. As not only the position $(x,y,z)$, but also the time difference between the GPS receiver's clock and the synchronized clocks of the satellites is unknown, four satellites are needed to obtain a ``fix''. Due to the way information from the satellites is coded, getting an initial fix can take on the order of minutes, but afterwards it becomes available multiple times per second. GPS measurements are neither precise nor accurate enough for robotics applications, and require to be combined with other sensors, such as IMUs. (Note that the bearing shown on some GPS receivers you may have access to is calculated from subsequent positions and is therefore meaningless if the robot is not moving.)

There exist also a variety of indoor GPS solutions, which consists of either active or passive beacons that are mounted in the environment at known locations. Passive beacons, for example infrared reflecting stickers arranged in a certain pattern or 2D barcodes, can be detected using cameras and their pose can be calculated from their known dimensions. Active beacons instead usually emit radio, ultrasound or a combination thereof, which can then be used to estimate the robot's range to this beacon. In this domain, ultra-wideband radio in particular is common for relative localization indoors.

\section*{Take-home lessons}
\begin{itemize}
\item Most of a robot's sensors either address the problem of determining the robot's pose or localizing and recognizing objects in its vicinity.
\item Each sensor has advantages and drawbacks that are quantified in its range, precision, accuracy, and bandwidth. Therefore, robust solutions to a problem can only be achieved by combining multiple sensors with differing operation principles.
\item Solid-state sensors (i.e.\ without mechanical parts) can be miniaturized and cheaply manufactured in quantity. This has enabled a series of affordable IMUs and 3D depth sensors that will provide the data basis for localization and object recognition on mass-market robotic systems.
\end{itemize}

\section*{Exercises}\small
\begin{enumerate}
\item Given a laser scanner with an angular resolution of 0.01 rad and a maximum range of 5.6 meters, what is the minimum range $d$ a robot needs to have from an object of 1cm width to definitely sense it, i.e., hit it with at least one of its rays? You can approximate the distance between two rays with the arc length.
\item Why does the bandwidth of a ultrasound based distance sensor decrease significantly when increasing its dynamic range, but that of a laser range scanner does not for typical operation?
\item You are designing an autonomous electric car to transport goods on campus. As you are worried about cost, you are thinking about whether to use a laser scanner or an ultrasound sensor for detecting obstacles. As you drive rather slow, you are required to sense up to 15 meters. The laser scanner you are considering can sense up to this range and has a bandwidth of 10Hz. Assume 300m/s for the speed of sound in the following.
\begin{enumerate}
\item Calculate the time it takes until you hear back from the US sensor when detecting an obstacle 15m away. Assume that the robot is not moving at this point.
\item Calculate the time it takes until you hear back from the laser scanner. Hint: you don’t need the speed of light for this, the answer is in the specs above.
\item Assume now that you are moving toward the obstacle. Which sensor will give you a measurement that is closer to your real distance at the time of reading and why?
\end{enumerate}
\item Pick an educational robot platform of your choice and make a list of its sensors.
\item Construct a simple range scanner by mounting an ultrasound sensor onto a servo motor. Implement a scanning routine that allows you to collect the raw data and display it on the screen. Can you see simple features such as corners and openings?
\item Explore the internet for do-it-yourself robotics shops. What kind of sensors do they offer? What are the interfaces these sensors provide?
\item Pick a physical sensor that you have access to. Can you design an experiment to characterize its precision and accuracy?
\item Your task is to design a sensor that can detect the remaining void in a parcel for an e-commerce application.
\begin{enumerate}
\item What sensors could you think off that would allow you to measure the volume of the content in the box?
\item What additional sensors could you use assuming the box is moving on a conveyor belt.
\item What additional information would you need to know in order to differentiate between box content, the box itself, and the environment around the box? What sensors could you use to get this information?
\item Additional sensors are not within your budget. What kind of measures could you take to reduce the amount of information required?
\end{enumerate}
\item Your task is to design an autonomous cart that can automatically dock with shelves in the environment.
\begin{enumerate}
\item What kind of sensors could you use to locate the shelf in the environment? Assume that the shelf is the only object in a certain target area.
\item What kind of physical measures could you take to simplify detection of the shelf?
\item  What kind of sensors could you use to detect whether the shelf is in a suitable position for docking?
\item What kind of physical measures could you take to simplify the sensing process?
\end{enumerate}
\item You are designing a competitive controller for the ``Ratslife'' game. What kind of information does the environment provide and what kind of sensor would you need to exploit it?
\end{enumerate}\normalsize


\part{Computation}
\chapter{Vision}\label{chap:vision}

Vision is one of the most information-rich sensor systems both humans and robots have available. However, efficiently and accurately processing the wealth of information that is generated by vision sensors is still a key challenge in the field. The goals of this chapter are to:
\begin{itemize}
\item introduce the concept of images as two-dimensional signals;
\item provide an intuition of the wealth of information hidden in low-level information;
\item introduce basic convolution and threshold-based image processing algorithms.
\end{itemize}

\section{Images as two-dimensional signals}

Images are captured by cameras containing matrices of charge-coupled devices (CCD) or similar semi-conductors (e.g. complementary metal–oxide semiconductor, CMOS) that can turn photons into electrical signals. These matrices can be read out pixel by pixel and turned into digital values, for example an array of 640 by 480 three-byte tuples corresponding to the red, green, and blue (RGB) components the camera has seen. An example of such data is shown in \cref{fig:iss_closeup}; for simplicity, we show only one color channel.
Looking at the data in the matrix clearly reveals the white tile within the black tiles at the lower-right corner of the chessboard. Higher values correspond to brighter colors (white) and lower values to darker colors. We also observe that although the tiles have to have the same color, the actual values differ quite a bit. It might make sense to think about these values much like we would do if the data would be 1D signal: taking the ``derivative'', e.g., along the horizontal rows, would indicate areas of big changes, whereas a ``frequency'' histogram of an image  would indicate how quickly values change. Areas with smooth gradients, e.g., black and white tiles, would then have low frequencies, whereas areas with strong gradients, would contain high frequency information.

\begin{figure}
    \centering
    \includegraphics[width=\textwidth]{figs/iss_closeupmatrix}
    \caption{A chessboard floating inside the ISS with astronaut Gregory Chamitoff. The inset shows a sample of the actual data recorded by the image sensor. One can clearly recognize the contours of the white tile.}
    \label{fig:iss_closeup}
\end{figure}

This language opens the door to a series of signal processing concepts, such as low-pass filters (suppressing high frequency information), high-pass filters (suppressing low frequency information), or band-pass filters (letting only a range of frequencies pass), analysis of the frequency spectrum of the image (the distribution of content at different frequencies), or ``convolving'' the image with another two-dimensional function. The next sections will provide both an intuition of what kind of meaningful information is hidden in such abstract data and provide concrete examples of signal processing techniques that make this information appear.

\section{From signals to information}

Unfortunately, many phenomena that often have very different or even opposite meaning look very similar when looking at the low-level signal. For example, drastic changes in color values do not necessarily mean that the color of a surface indeed has changed. Similar patterns are generated by depth discontinuities, specular highlights, changing lighting conditions, or surface orientation changes. These phenomena---some of which are illustrated in \cref{fig:iss_edges}---make computer vision a hard problem.

\begin{figure}[!htb]
    \centering
    \includegraphics[width=\textwidth]{figs/iss_edges}
    \caption{Inside of the international space station (left), circled areas in which pixel values change drastically (right). Underlying effects that produce similar responses: change in surface properties (1), depth discontinuities (2), specular highlights (3), changing lighting conditions such as shadows (4), or surface orientation changes (5).
    \label{fig:iss_edges}}
\end{figure}

This example illustrates that signals and data alone are not sufficient to understand a phenomenon, but require context. Here, the context does not only refer to surrounding signals, but also high-level conceptional knowledge such as the fact that light sources create shadows and specular highlights, that objects in the front appear larger, and so on. The importance of such conceptional knowledge is illustrated in \cref{fig:craters}:
both images show an identical landscape that once appears to be speckled with craters, once with bubble-like hills. At first glance, both scenes are illuminated from the left, suggesting a change in the landscape. However, once information that the sun is illuminating one picture from the left and the other from the right is available, the paradox becomes clear: the variable illumination makes the craters look like bumps under come conditions.

\begin{figure}[!htb]
    \centering
    \includegraphics[width=\textwidth]{figs/craters}
    \caption{Picture of the Apollo 15 landing site during different times of the day. The landscape is identical, but appears to be either speckled with craters (left) or hills (right). Knowing that the sun is illuminating the scene from the left and right, respectively, does explain this effect. Image credit: NASA/GSFC/Arizona State University.
    \label{fig:craters}}
\end{figure}

More surprisingly, conceptual knowledge is often sufficient to make up for the lack of low-level cues in an image. An example is shown in \cref{fig:dalmatian}. Here, a Dalmatian dog can be clearly recognized despite the absence of cues for its outline, i.e.\ by simply extrapolating its appearance and pose from conceptual knowledge.

These examples illustrate both the advantages and drawbacks of a signal processing approach to computer vision. While an algorithm will detect interesting signals even there where we don't see or expect them (due to conceptional bias), image understanding not only requires low-level processing, but also intelligent combination of the spatial relationship between low-level cues and conceptual knowledge about the world. As we will later see (\cref{chap:ann}), this can be accomplished through convolutional neural networks that provide a single pipeline to process information at different scales---ranging from extracting local features to examining their spatial relationships with each other.

\begin{figure}[!]
    \centering
    \includegraphics[width=\textwidth]{figs/dalmatian}
    \caption{The image of a Dalmatian dog can be clearly recognized by most spectators even though low-level cues such as edges are only present for ears, chin and parts of the legs. The contours of the animals are highlighted in a flipped version of the image in the inset.
    \label{fig:dalmatian}}
\end{figure}

\section{Basic image operations}

Basic image operations can be thought of as a filter that operates in the frequency or in the space (intensity/color) domain. Although most filters directly operate in the intensity domain, knowing how they affect the frequency domain is helpful in understanding the filter's function. For example, a filter that is supposed to highlight edges such as the one shown in \cref{fig:iss_edges} should suppress low frequencies---i.e., areas in which the color values do not change much, and amplify high-frequency information---i.e., areas in which the color values change quickly. The goal of this section is to provide a basic understanding of how basic image processing operation works. It is important to note that the methods presented here, while still valid, have been superseded by more sophisticated implementations that are widely available as software packages or within desktop graphic software.

\subsection{Threshold-based operations}
In order to find objects with a certain color or edge amplitude, thresholding an image will lead to a binary image that contains ``true-false'' regions that fit the desired criteria. Thresholds make use of operators like $>,<,\leq,\geq$ and combinations thereof. There also exist adaptive versions that adapt/update the thresholds locally, e.g., to make up for changing lighting conditions.
Albeit thresholding is simple if compared to other techniques shown below, finding correct threshold values is a hard problem. In particular, actual pixel values change drastically with change in the lighting conditions and there is no such thing as ``red'' or ``green'' when inspecting the actual values under different conditions.

\subsection{Convolution-based filters}

\screencast{https://commons.wikimedia.org/wiki/File:Convolution_of_box_signal_with_itself2.gif}{convolution}

A filter can be implemented using the \textsl{convolution}\index{Convolution} operator $\star$ that convolves function $f()$ with function $g()$:
\begin{equation}
f(x)\star g(x)=\int_{-\infty}^{\infty}f(\tau)g(x-\tau)d\tau \quad,
\end{equation}
where $g()$ is defined as \textsl{filter}\index{Filter}.
The convolution essentially ``shifts'' the function $g()$ across the function $f()$ while multiplying the two (see also in the video to the left). As images are discrete signals, the convolution is consequently discrete:
\begin{equation}
f[x]\star g[x]=\sum_{i=-\infty}^{\infty}f[i]g[x-i]\quad .
\end{equation}
Additionally, given that images are two-dimensional signals, the convolution is two-dimensional as well:
\begin{equation}\label{eq:2dconv1}
f[x,y]\star g[x,y]=\sum_{i=-\infty}^{\infty}\sum_{j=-\infty}^{\infty}f[i,j]g[x-i,y-j]\quad.
\end{equation}
Although we have defined the convolution from negative infinity to infinity, both images and filters are usually finite. Images are constrained by their resolution, and filters are usually much smaller than the images themselves. Also, the convolution is commutative, therefore \cref{eq:2dconv1} is equivalent to:
\begin{equation}\label{eq:2dconv2}
f[x,y]\star g[x,y]=\sum_{i=-\infty}^{\infty}\sum_{j=-\infty}^{\infty}f[x-i,y-j]g[i,j].
\end{equation}

\subsubsection{Gaussian smoothing}

One of the most basic (and important) filters is the \textsl{Gaussian filter}\index{Gaussian filter}. The Gaussian filter is shaped like the Gaussian bell function and can be easily stored in a two-dimensional matrix. Implementing a Gaussian filter is surprisingly simple, e.g.:
\begin{equation}
g(x,y)=\frac{1}{10}\quad.
\left(
\begin{array}{ccc}
1 & 1 & 1\\
1 & 2 & 1\\
1 & 1 & 1\\
\end{array}
\right)
\end{equation}
Using this filter in Equation~\ref{eq:2dconv2} on an infinitely large image $f()$ leads to
\begin{equation}\label{eq:2dconv3}
f[x,y]\star g[x,y]=\sum_{i=-1}^{1}\sum_{j=-1}^{1}f[x-i,y-j]g[i,j]\quad,
\end{equation}
assuming that $g(0,0)$ is the center of the matrix. What now happens is that each pixel $f(x,y)$ becomes the average of that of its neighbors, with its previous value weighted twice than that of its neighbors (because $g(0,0)=0.2$). More concretely:
\begin{equation}
f(x,y)=
\begin{smallmatrix*}[l]
f(x+1,y+1)g(-1,-1) &+f(x+1,y)g(-1,0) &+f(x+1,y-1)g(-1,1)\\
+f(x,y+1)g(0,-1) &+f(x,y)g(0,0) &+f(x,y-1)g(0,1)\\
+f(x-1,y+1)g(1,-1) &+f(x-1,y)g(1,0) &+f(x-1,y-1)g(1,1)
\end{smallmatrix*}
\end{equation}
Doing this for all $x$ and all $y$ is equivalent to physically ``sliding'' the filter $g()$ along the image.

\begin{figure}[!]
    \centering
    \includegraphics[width=\textwidth]{figs/filters}
    \caption{A noisy image before (top left) and after filtering with a Gaussian kernel (top right). Corresponding edge images are shown underneath.
    \label{fig:filters}}
\end{figure}

An example of the Gaussian filter in action is shown in \cref{fig:filters}. The filter acts as a \textsl{low-pass filter}\index{Low-pass filter}, suppressing high frequency components. Indeed, noise in the image is suppressed, leading also to a smoother edge image, which is shown to the right.

\subsubsection{Edge detection}\label{sec:sobel}

Edge detection can be achieved using another convolution-based filter, the \textsl{Sobel} kernel\index{Sobel filter}:
\begin{equation}
s_x(x,y)=
\left(
\begin{array}{ccc}
-1 & 0 & 1\\
-2 & 0 & 2\\
-1 & 0 & 1\\
\end{array}
\right)
\qquad
s_y(x,y)=
\left(
\begin{array}{ccc}
1 & 2 & 1\\
0 & 0 & 0\\
-1 & -2 & -1\\
\end{array}
\right)
\end{equation}
Here, $s_x(x,y)$ can be used to detect vertical edges, whereas $s_y(x,y)$ highlights horizontal edges. Edge detectors such as the \textsl{Canny} edge detector\index{Canny edge detector} therefore run at least two of such filters over an image to detect both horizontal and vertical edges.

\subsubsection{Difference of Gaussians}
\label{sec:vision:dog}

An alternative method for detecting edges is the \textsl{Difference of Gaussians} (DoG) method\index{Difference of Gaussians (DoG)}. The idea is to subtract two images that have each been filtered using a Gaussian kernel with different width. Both filters supress high-frequency information and their difference therefore leads to a \textsl{band-pass} filtered signal\index{Band-pass filter}, from which both low and high frequencies have been removed. As such, a DoG filter acts as a capable edge detection algorithm. Here, one kernel is usually four to five times wider than the other, therefore acting as a much stronger filter.

Differences of Gaussians can also be used to approximate the \textsl{Laplacian of Gaussian}\index{Laplacian of Gaussian}, i.e., the sum of the second derivatives of a Gaussian kernel. Here, one kernel is roughly 1.6 times wider than the other. The band-pass characteristic of DoG and LoGs are important as they highlight high-frequency information such as edges, yet suppress high-frequency noise in the image.

\subsection{Morphological Operations}

Another class of filters are morphological operators which consist of a kernel describing the structure of the operation (this can be as simple as an identity matrix) and a rule on how to change a pixel value based on the values in the neighborhood defined by the kernel.
%
Important morphological operators are \textsl{erosion} and \textsl{dilation}\index{Erosion}\index{Dilation}. The erosion operator assigns a pixel a value with the minimum value that it can find in the neighborhood defined by the kernel. The dilation operator assigns a pixel a value with the maximum value it can find in the neighborhood defined by the kernel. This is useful, e.g., to fill holes in a line or remove noise. A dilation followed by an erosion is known as a ``Closing'' and an erosion followed by a dilation as an ``Opening''. Subtracting erosed and dilated images from each other can also serve as an edge detector. Examples of such operators are shown in \cref{fig:morphology}.

\begin{figure}
    \centering
    \includegraphics[width=\textwidth]{figs/morphology}
    \caption{Examples of morphological operators erosion and dilation and combinations thereof (image credit: OpenCV documentation, BSD).
    \label{fig:morphology}}
\end{figure}

\section{Extracting Structure from Vision}\label{sec:vision:structure}

A remarkable property of vision is the ability to provide both semantic (\textsl{qualities} of the scene, such as what is in it) and metric (\textsl{quantities} of the scene, such as sizes and distances) information. Extracting semantic information is nowadays heavily reliant on machine learning, which is explained at a high level in \cref{sec:cvml}. The extraction of metric information however can be accomplished by leveraging geometric relationships, which we will describe here.

\cref{fig:stereovision} shows a high-level schematic of the relationships between an image frame and another---both observing the same point. In here, we do not draw a distinction between these two frames being either spatially or temporally correlated, which are two distinct problems in robotics: in \textsl{stereo vision}\index{Stereo Vision}, two cameras are rigidly attached to one another (spatial correlation) and are acquiring images of the same scene; in \textsl{structure from motion}\index{Structure From Motion}, a single camera is moved through a scene and a pair of images from the single camera are related to one another via a transform matrix (temporal correlation).
In either case, it is possible to identify the ``projection center'' of the camera frames as $C_L$ and $C_R$; they related to one another through $T_{LR}$, which is defined as the transform matrix from the left to the right frame. In stereo vision, this transform is known as the \textsl{sensor extrinsics}, a 6-dof quantity that must be estimated through calibration. In structure from motion, this transform quantifies the motion applied to the camera, which can be estimated through localization (see \cref{chap:localization}).

\begin{figure}
\centering
    \def\svgwidth{1.0\textwidth} 
    \import{./figs/}{stereovision.pdf_tex}
\caption{Schematic of correlating features across images in order to extract three-dimensional information from two-dimensional views.}
\label{fig:stereovision}
\end{figure}

Note that since the camera pair takes two images of the same scene, two projections into the corresponding image planes of the same point in the world can be correlated with one another to determine the point's 3D position. This may be accomplished naively through identifying the point $p_{P_L}$ in $C_L$ and searching for that point $p_{P_R}$ in $C_R$. Crucially, the nature of projections of 3D points into the camera frame is a known operation, and in fact a very simple one. In particular, a 3D point in the world can be projected into the camera frame using:
\screencast{https://youtu.be/ND2fa08vxkY}{cameraparameters}
\begin{equation}
\begin{pmatrix}
u\\
v\\
1
\end{pmatrix}
= K T \begin{pmatrix}
x\\
y\\
z\\
1 
\end{pmatrix}
\label{eq:stereovision}
\end{equation}

\noindent where $K$ is known as the \textsl{camera intrinsic matrix} and $T$ is the matrix form of the transform between the camera and some global coordinates in which the point $(x, y, z)$ is expressed. Note that the \textsl{camera intrinsic matrix} is another $\textsl{calibrated}$ quantity, instantiated by two optical center parameters and two scaling parameters.
Importantly, it is possible for two projected points (representing a single point in 3D space) to be calculated directly through triangulation on these 2D point pairs in images alone. Using the same math as in Eq.\ \eqref{eq:stereovision} but expressing $C_L$ as the global coordinate system, we can relate the 3D coordinate of the point to the two 2D measurements, camera intrinsic matrix, and $T_{LR}$:

\begin{equation}
\begin{pmatrix}
u_R\\
v_R\\
1
\end{pmatrix}
= K T_{LR} \begin{pmatrix} x \\ y \\ z \\ 1\end{pmatrix}
= K T_{LR} K^{-1} \begin{pmatrix} u_L \\ v_L \\ 1 \end{pmatrix}
\end{equation}

Note that this expression is frequently given in terms of what is known as the ``essential matrix,'' which is nominally a technique to solve this problem for uncalibrated cameras. This expression, clearly induced by the geometry expressed in \cref{fig:stereovision}, allows for alternatively solving of the values making up the entries of the essential matrix and not those of the camera intrinsic and extrinsic parameters.

Looking back at the geometry in \cref{fig:stereovision}, it may be noted that $p_{P_L}$ lies on a line that extends from the center of camera $C_L$ to the point $p$. However, there is ambiguity of the depth along ray that is cast from $C_L$ to $p_c$ $\overrightarrow{C_L p_c}$. In order to disambiguate this depth, the line between $p_{P_L}$ and the center of projection of $P_L$, which is known as the ``epipolar line,'' \index{Epipolar Line} may be projected into $P_R$, which creates the so-called ``epipolar line of $p_L$ in $C_R$.'' It is along this line that the point $p_c$ projected into $P_R$ may be found. Most notably from this is that therefore the search for the point $p_{P_R}$ may be reduced to a line search along the projected epipolar line. This line search is much more efficient than finding the projection of point $p_c$ in the whole image plane of $P_R$, and therefore allows for rapidly sped-up geometry calculations across image pairs.

%There is no single way to approach the problem of metric information extraction from images, but these techniques should provide some insight into the general manipulations executed in order to extract these data. With a little geometry, there is clearly much more than meets the eye in images than objects and classes.

%We will begin by explaining the motivation behind image coordinates. We note in \cref{fig:imagecoordinates} that in a perfect ``linear'' camera, a particular pixel in an image is given coordinates $(u,v)$ from the center point of the image. That is, the center point is denoted the origin $(0,0)$; this differs from standard image encodings, where the ``top left'' of the image is the origin. Frequently this requires $(c_x, c_y)$.

Extracting metric information from images requires to uniquely identifying identical points in each image. A simple solution to this problem is what is known as \textsl{structured light}\index{Structured light} and is illustrated in \cref{fig:struclight}.
%
Thanks to the continuously increasing efficiency of computational systems, a light-weight version of such an approach has become feasible to be implemented at small scale and low cost around $2010$, and emerged as a novel standard in robotic sensing.

\begin{figure}
	\centering
		\includegraphics[width=\textwidth]{figs/structuredlight.png}
	\caption{From left to right: two complex physical objects, a pattern of colored straight lines and their deformation when hitting the surfaces, reconstructed 3D shape. From \protect\cite{zhang2002rapid}.}
	\label{fig:struclight}
\end{figure}

Instead of using line patterns, infrared-based depth image sensors use a speckle pattern (a collection of randomly distributed dots with varying distances).
%, and two computer vision concepts: \textsl{depth from focus} and \textsl{depth from stereo}.\index{Depth from Focus}\index{Depth from Stereo} When using a lens with a narrow focal depth, objects that are closer or farther away appear blurred; you can easily observe this on professional portrait photos, which often use this effect for aesthetic purposes under the name of ``bokeh''.
%Measuring the ``blurriness'' of a scene (for known camera parameters) therefore allows an initial estimate of depth.
%Conversely, depth from stereo works by measuring the disparity of the same object appearing in two images taken by cameras that are a known distance apart. Being able to identify the same object in both frames allows to calculate this disparity, and from there the distance of the object: the farther the object is, the smaller the disparity will be.
Identifying identical points in two images simply requires to search for blobs with similar size that are close to each other.


\section{Computer Vision and Machine Learning}
\label{sec:cvml}
The algorithms described here still form the basis of most image understanding pipelines and make feature detection (\cref{chap:feature_extraction}) tractable. With the advent of so-called ``convolutional neural networks'' (\cref{chap:ann}), basic signal processing such as described here is now often wrapped into the image understanding problem. While this makes it less important to implement such algorithms oneself, understanding what convolution, morphological operations and thresholds do to visual information remains still relevant to meaningfully compose neural networks and make them less of a black box.


\section*{Take-home lessons}
\begin{enumerate}
\item Unlike the sensors from \cref{chap:sensors}, our brains can directly process the 2D information that is captured by a vision sensor. It is difficult to unthink the amount of processing that we perform automatically, augmenting the signal with knowledge and other information that the computer does not necessarily have.
\item  Algorithms described in this chapter aim at reducing information to a lower-dimensional space by removing noise and other spurious information, making the related challenge of understanding the data more tractable. 
\item There is a trade-off between making the data stream more tractable and preserving actual information. As computers and algorithms, in particular machine learning, become more powerful, modern vision systems often blend pre-processing and actual image understanding into a single pipeline. 
\end{enumerate}


\section*{Exercises}\small
\begin{enumerate}
\item Below are shown multiple ``Kernels'' that can be used for convolution-based image filtering.
\begin{equation}
\nonumber
\begin{array}{|c|c|c|}
\hline
1 & 1 & 1\\
\hline
1 & 2 & 1\\
\hline
1 & 1 & 1\\
\hline
\end{array}
\quad
\begin{array}{|c|c|c|}
\hline
0 & -1 & 0\\
\hline
0 & -1 & 0\\
\hline
0 & -1 & 0\\
\hline
\end{array}
\quad
\begin{array}{|c|c|c|}
\hline
1 & 1 & 1\\
\hline
1 & -4 & 1\\
\hline
1 & 1 & 1\\
\hline
\end{array}
\end{equation}
\begin{enumerate}
\item Identify the Kernel, which can blur an image.
\item What kind of features can be detected by the other two kernels?
\end{enumerate}
\item How many for-loops are needed to implement a 2D convolution? Explain your reasoning.
\item Use an appropriate robot simulation environment that allows you access to a simulated camera in a world with simple features such as geometric shapes of different color etc.
\begin{enumerate}
\item Implement a thresholding algorithm that allows you to black out anything but an object of a specific color. Is a simple threshold enough? Why not? Can you black out an object using a low and and a high threshold?
\item Implement a smoothening algorithm by performing both a convolution with a Gaussian kernel as well as a series of morphological operations. Experiment with kernels of different width and different steepness. What are the advantages and drawbacks of using morphological operations over a simple Gaussian filter?
\item Implement an edge detection algorithm, e.g. by performing a convolution with a Sobel kernel. Experiment with different kernels. What else do you need to do to create an image that only contains edges?
\end{enumerate}
\item Can you think about a smoothening algorithm that will only smoothen small amounts of noise, but maintains edges? What kind of filtering algorithms could you combine to achieve this goal?
\item Explore the internet for a computer vision toolbox that supports your language of choice. What do you find? Does the toolbox implement all of the algorithms in this chapter? Solve the above assignments using the toolbox's built-in functions.
\item Use an appropriate robot simulation environment that allows you to simulate two cameras that are at a known distance in the same plane. Use simple geometric objects such as a red ball and compute their distance using stereo disparity. 
\end{enumerate} \normalsize

\input{chapters/features}
%!TEX root = ../book.tex
\chapter{Artificial Neural Networks}\label{chap:ann}

\textsl{Artificial neural networks (ANNs)}\index{Artificial Neural Networks} are part of a class of machine learning techniques that are loosely inspired by neural operation in the human brain; in robotics, they are generally used to classify or regress data for the dual purposes of perception (e.g., \cref{chap:vision,chap:feature_extraction}) and control (as will be shown in \cref{chap:taskexecution}).
%
While ANNs for the longest time have been just one of the many methods available to roboticists from the neighboring field of machine learning, recent advances in computing---in particular, graphical processing units (GPU)---and the availability of large datasets have enabled the training of neural networks with many layers, commonly referred to as \textsl{deep learning}\index{Deep Learning}. These (often massive) networks have led to revolutionary results in many fields including computer vision, natural language processing, video and speech processing, and robotics.
%
Not too long ago, neural networks were considered ``deep'' if the had just more than two layers. Today, ``deep'' neural networks can have hundreds of layers and thousands of inputs and outputs---or more!
This is still shy of the human brain, which contains $\sim 100^{11}$ neurons, each with thousands of synapses connecting a single neuron to thousands of others.

\begin{mdframed}
Remember: a \textsl{classification}\index{Classification (neural networks)} problem requires that input data be classified between two or more classes; a \textsl{regression}\index{Regression (neural networks)} problem requires a prediction of a (possibly continuous or high-dimensional) quantity. While a regression problem can be converted into a classification one (and vice versa), the machine learning community generally considers them two separate applications, and different techniques are developed to perform in each of these domains.
\end{mdframed}

Machine learning is a large field that shares many of its foundations with robotics, in particular for what concerns probability theory and statistics. Deep learning may be used as a drop-in replacement for sensor pre-processing and conditioning, computer vision and feature extraction, localization, and even replace controllers for locomotion and grasping.
For each of these applications, it is important to understand when deep learning may perform better than traditional approaches, and when it does not.
In a nutshell, deep learning models become first choice when not enough information exists to model a system using first principles. While a ``deep enough'' model with the right architecture might approximate any existing function in robotics, deep learning models lack ``explainability'' beyond statistical accuracy, that is, we may not easily be able to know how the approach actually works (in terms of which criteria it uses for its decisions) and when it might fail, usually making it a second choice behind an approach based on first principles with clear decision-making rationale.

The goals of this chapter are to introduce:

\begin{itemize}
    \item basic neural networks from the simple perceptron to multi-layer neural networks,
    \item different network architectures and encodings to tackle a variety of regression and classification tasks,
    \item convolutional neural networks---including padding, striding, pooling and flattening, and how they can be used to process spatial and temporal data,
    \item recurrent neural networks that introduce memory for classifying temporal data and perform control tasks.
\end{itemize}


\section{The simple Perceptron}

\begin{figure}[!b]
    \centering
    % \includegraphics[width=0.5\columnwidth]{figs/linearlyseparable}
    \def\svgwidth{0.6\textwidth}
    \import{./figs/}{linearlyseparable.pdf_tex}
    \caption{A 2-dimensional dataset, where every element has two values ($x_1$ and $x_2$) and belongs to one of two classes (red and blue). In the simplest case of linear separation, it is possible to separate the two classes with just a straight line.\label{fig:linearsep}}
\end{figure}

Artificial neural networks are inspired by neurons and synapses in the human brain and have been studied since the Fifties. One of the earliest models is the \textsl{Perceptron}\index{Perceptron}, which can classify an input vector $x$ of dimension $m$ into two classes. Such a problem is shown in \cref{fig:linearsep}. Variations of the simple perceptron remain the basic elements of deep neural networks until today.
%
As detailed in \cref{fig:perceptron}, a perceptron has $m$ inputs $x_1$ to $x_m$)---each modulated by a weight $w_1$ to $w_m$, as well as a threshold $b$; it outputs either zero or one.

\begin{figure}
    \centering
    % \includegraphics[width=0.5\columnwidth]{figs/perceptron.png}
    \def\svgwidth{0.66\textwidth}
    \import{./figs/}{perceptron.pdf_tex}
    \caption{The simple perceptron passes the dot product between the inputs $x$ and weights $w$ through a Heaviside function, returning 1 when $wx+b>0$ and 0 otherwise.}\label{fig:perceptron}
\end{figure}

The perceptron classifies whether $x$ lies above or below a hyperplane defined by the weights $w=\{w_1, \ldots, w_m\}$ using the following equation:
%
\begin{equation} \label{eq:heaviside}
f(x)=\begin{cases}
1 \qquad wx+b > 0\\
0 \qquad otherwise
\end{cases}
\end{equation}
%
Here, $wx=\sum_{i=1}^mw_ix_i$ is the dot product and the non-linear activation function $f(x)$ is also known as \index{Heaviside step function}\textsl{Heaviside step function}. In practice, we are appending the value of '1' to the vector $x$ so that $x_0=1$, which simplifies $wx+b$ (with $w=\{w_1, \ldots, w_m\})$ to $wx$ with $w=\{w_0, w_1, \ldots, w_m\}$ where $w_0$ takes the role of $b$. This is illustrated in \cref{fig:perceptron}, where the bias $b$ is alternatively labeled by $w_0$ and input $x_0=1$.

\subsection{Geometric interpretation of the simple perceptron}

If $w$ really defines an hyperplane, we should be able to easily visualize it when $m=2$. When $m=2$, i.e. every data point $x$ has only two dimensions, the separating hyperplane is a line such as the one shown in \cref{fig:linearsep}.
Indeed, we can easily demonstrate this. Writing the dot product out yields:
\begin{equation}
w_1x_1+w_2x_2+b=0
\end{equation}
As we plot $x_1$ along the x-axis and $x_2$ along the y-axis, we can write:
\begin{equation}
w_1x+w_2y+b=0
\end{equation}
This can be rewritten into:
\begin{equation}
y=-\frac{w_1}{w_2}x-\frac{b}{w_2}\quad ,
\end{equation}
and displayed within a scatter plot.

\subsection{Training the simple perceptron}

Training the perceptron, equates to finding appropriate values for $w$ and $b$ that separate the data into two classes. This process can be performed iteratively:

\begin{enumerate}
\item initialize all the weights with zeros or a small random number;
\item compute the prediction $y_j=f(wx_j+b)$ for each data point $x_j$. A suitable choice for $f()$ is the Heaviside step function , e.g. (\cref{eq:heaviside});
\item calculate the mismatch between prediction $y_j$ and the true class $d_j$ to update the weights:
\begin{equation}
w(t+1)=w(t)+r(d_j-y_j)*x_j
\end{equation}
\item repeat steps $2$ and $3$ until a termination criteria, e.g. a decreasing error or maximum number of iterations, is reached.
\end{enumerate}

Albeit very simple, this learning algorithm has still a lot in common with state-of-the-art algorithms. First, weights are updated in an iterative process using small increments governed by the parameter $r$, which is referred to as the \textsl{learning rate}\index{Learning Rate}. By changing $w$ in small increments, the algorithm is literally rotating and translating the separating line in a direction that minimizes the \textsl{loss}, given by $d_i-y_i$. One can easily see that if the learning rate is too low, the algorithm will never find a good solution. One can also see that if the learning rate is too large, the line might move too much, ``skipping'' the configuration that achieves optimal separation.

It is worth noting that this simple implementation is a \textsl{de facto} implementation of gradient descent\index{Gradient descent}---in this case with a loss function of the form $(d_i-y_i)^2$, that can be minimized by moving against the direction of its gradient, here $2(d_i-y_i)$. Other examples of gradient descent can be found in \cref{sec:kinematics:inverse:feedbackcontrol}.

Second, the learning algorithm requires multiple presentations of the data-set, as the error is computed for every point in the data set. The more the amount of data, the longer the training! In this case the increase in time is linear---which is also generally true for more complex and modern learning algorithms.

Third, the error between the prediction and the true class is only calculated based on the given training data. Even if we were to train with unlimited amounts of data points, it would still be difficult to generalizes for new data, and whether these new measurements will be distributed in a way that is representative of the training data.

\section{Activation Functions}\label{sec:activationfunctions}

Using a on-off Heaviside step function makes training a neural network using gradient descent rather difficult, as a function that switches from ``not working at all'' to ``working completely'' provides very little information in which direction to move. It is therefore more desirable to have a smoother activation function. One such function is the \index{Sigmoid function}\textsl{sigmoid function}:
\begin{equation}
\sigma(x)=\frac{1}{1+e^{-x}}
\end{equation}
Its main characteristics are that it asymptotically stays between $0$ and $1$, and crosses the y-axis at $0.5$. It is shown in \cref{fig:activationfunctions}, left.

\begin{figure}[htb]
    \centering
    % \includegraphics[width=0.9\columnwidth]{figs/activationfunctions}
    \def\svgwidth{0.9\textwidth}
    \import{./figs/}{activationfunctions.pdf_tex}
    \caption{Typical activation functions used in neural networks: the sigmoid activation function, left, and the rectified linear unit (ReLU), right.\label{fig:activationfunctions}}
\end{figure}

The sigmoid function is attractive for learning as the direction in which the weights should move to improve the error is very clear in the vicinity of $wx=0$, and computing its derivative is rather simple. While attractive in many cases, the sigmoid function has some drawbacks. For instance, when $wx$ is very large or very small, the neuron either saturates or never activates---a phenomenon known as the \textsl{vanishing gradient} problem. Another drawback is that computing the sigmoid function is computationally expensive. An alternative is the hyperbolic tangent $\tanh()$ which remains in the range of -1 to 1 and crosses the y-axis at 0.

A popular solution to decrease computation time is the \textsl{Rectified Linear Unit (ReLU)}\index{Rectified Linear Unit (ReLU)}, which is given by:
\begin{equation}
R(x)=max(0,x)
\end{equation}
and is shown in \cref{fig:activationfunctions}, right.
The dashed line indicates a refinement of the ReLU known as \textsl{leaky ReLU} with a typical slope of $0.1$; it improves learning for negative $wx$ by providing a directional gradient.

Please note that we only talk about ``perceptrons'' when the Heaviside step function is used as activation function.

\section{From the simple perceptron to Multi-layer neural networks}

We have seen that the single perceptron is able to linearly separate a dataset, returning ``$0$'' or ``$1$'' as a function of the data being below or above the separating hyperplane defined by the weight vector $w$. However, it is easy to see that some problems cannot be linearly separated.
%
\begin{figure}[htb]
    \centering
    % \includegraphics[width=0.9\columnwidth]{figs/xorproblem}
    \def\svgwidth{0.9\textwidth}
    \import{./figs/}{xorproblem.pdf_tex}
    \caption{Data that cannot be separated using a single line (left) in canonical form (center). This problem is known as the ``XOR'' problem due to the truth table of the associated classification problem (right).\label{fig:xorproblem}}
\end{figure}
%
In the example shown in \cref{fig:xorproblem}, the red and the blue data points are not separable by a single line, but require at least two lines. This problem is known as the ``XOR'' problem, which can be seen by looking at just four data points at $(0,0)$, $(0,1)$, $(1,0)$, and $(1,1)$. Tabulating this data together with its color, reveals a truth table with the characteristics of logical exclusive or (XOR), i.e. $x_1$ and $x_2$ have to be different for the output to be true (here ``blue''), whereas the output is false (here ``red'') when the inputs are the same.

We already know that a single perceptron can create a single separating hyperplane; we will therefore need at least two perceptrons to solve the XOR problem.
Using two perceptrons in parallel will yield us with tuples of the kind $(0,0)$, $(0,1)$ and so on; hence, we then need another perceptron to recombine these tuples into a single output.
\cref{fig:basicmultilayer} shows the simplest multi-layer perceptron that can be trained for the XOR problem, with one \index{Input Layer}\textsl{input layer}, a so-called\index{Hidden Layer} \textsl{hidden layer}, and an \index{Output Layer}\textsl{output layer}.

\begin{figure}
    \centering
    % \includegraphics[width=0.8\columnwidth]{figs/basicmultilayernetwork}
    \def\svgwidth{0.8\textwidth}
    \import{./figs/}{basicmultilayernetwork.pdf_tex}
    \caption{A simple multi-layer perceptron with one input layer, one hidden layer, and one output layer. \label{fig:basicmultilayer}}
\end{figure}

\subsection{Formal description of Artificial Neural Networks}

As with the simple perceptron, we will use node $i$'s bias as the $0-$th weight vector, that is:
\begin{equation}
w^k_{0,j}=b^k_j
\end{equation}
Here, we use the following notation: we denote the layer with a superscript, and the index of the incoming node and the outgoing node with a subscript tuple.
That is, $w^k_{i,j}$ is connecting the $i-$th incoming weight to the $j-$th node of the $k-$th layer (the $i-$th incoming weight is the $j-$th node in layer $k-1$). This, as well as the simple example network above, is illustrated in \cref{fig:backpropnotation}.
Each layer, denoted by the index $k$, has exactly $r^k$ nodes.
\begin{figure}[htb]
    \centering
    % \includegraphics[width=0.7\columnwidth]{figs/backpropnotation}
    \def\svgwidth{0.7\textwidth}
    \import{./figs/}{backpropnotation.pdf_tex}
    \caption{Notation used to index weights (left) with respect to layer $k$ and the multi-layer network from \cref{fig:basicmultilayer} (right).\label{fig:backpropnotation}}
\end{figure}

\subsubsection{Inputs and outputs}

The output $o_i$ of node $i$ is given by:
\begin{equation}
o_i=g(a_i^k)\quad,
\end{equation}
where $g()$ is a non-linear activation function such as---but not limited to---the ones described in \cref{sec:activationfunctions} or the Heaviside step function. Here, $a_i^k$ is known as the \textsl{activation}\index{Activation (neural network)}, i.e. the weighted sum computed by node $i$ in layer $k$:
\begin{equation}
a_i^k=\sum_{j=0}^{r_{k-1}}w_{j,i}^ko_j^{k-1}\quad,
\end{equation}
with $o_j^{k-1}$ the $j-$th output of the previous layer. This is illustrated in \cref{fig:backpropnotation2}.

\begin{figure}[htb]
    \centering
    % \includegraphics[width=0.7\columnwidth]{figs/backpropnotation2}
    \def\svgwidth{0.7\textwidth}
    \import{./figs/}{backpropnotation2.pdf_tex}
    \caption{Inputs and outputs of neuron $i$ in the $k$-th layer showing activation $a_i^k$ and output $o_i^k$.\label{fig:backpropnotation2}}
\end{figure}

In case of $k$ being the output layer, $o_i^k$ should be equivalent to $y_i^k$. Likewise, in case of $k-1$ being the input layer $o_i^{k-1}=x_i$.

\subsection{Training a multi-layer neural network}

Finding a set of weights and bias values, that is few parameters for a simple two-dimensional problem but potentially billions for a ``deep network'', is an NP-complete problem \cite{blum1992training}.
We therefore need an efficient approximation. To this end, we consider a training datasets consisting of input-output pairs $x_i$ and $y_i$ with $i=1..N$, and a feed-forward neural network with parameters $w$.

\subsubsection{Loss function}\label{sec:lossfunction}

The goal of training is to minimize an error function such as the mean squared error:
\begin{equation}
E(x,y,w)=\frac{1}{2N}\sum_{i=1}^{N}(\hat{y_i}-y_i)^2
\end{equation}
between the output $\hat{y_i}$ that the neural network with parameters $w$ computes and the known value $y_i$ from the \index{Training set}\textsl{training set} is minimized.
%
Similar to the perceptron, we can reduce $E(x,y,w)$ by iteratively descending along its gradient, i.e.:
\begin{equation}
w(t+1)=w(t)-\alpha \frac{\partial E(x,y,w(t))}{\partial w}
\end{equation}

This process is non-trivial, as calculating the partial derivatives across the computation graph of the neural network requires the chain rule. An algorithm known as \textsl{Backpropagation} is described in \cref{ch:backpropagation}.


\section{From single outputs to higher dimensional data}

Extending a neural network from one single output to multiple binary classifiers is straightforward, requiring only to increase the dimensionality of the output vector.
Much less straightforward is encoding more complex data which leads to the following question: how can we represent numerical values, such as digits from $0$ to $9$ or characters from A to Z?

\subsubsection{One-Hot Encoding}
A very common approach is known as \textsl{One-Hot Encoding (OHE)}. In OHE, $n$ discrete labels such as numbers or characters are encoded as a binary vector of length $n$. To encode the $i-$th element of a set of labels, this vector is zero except at position $i$. For example, to encode the characters $0 \ldots 9$, OHE would represent them as:

\begin{eqnarray}
\nonumber
0 &=& (1,0,0,0,0,0,0,0,0,0)\\
\nonumber
1 &=& (0,1,0,0,0,0,0,0,0,0)\\
\nonumber
2 &=& (0,0,1,0,0,0,0,0,0,0)\\
\nonumber
3 &=& (0,0,0,1,0,0,0,0,0,0)\\
\nonumber
4 &=& (0,0,0,0,1,0,0,0,0,0)\\
\nonumber
5 &=& (0,0,0,0,0,1,0,0,0,0)\\
\nonumber
6 &=& (0,0,0,0,0,0,1,0,0,0)\\
\nonumber
7 &=& (0,0,0,0,0,0,0,1,0,0)\\
\nonumber
8 &=& (0,0,0,0,0,0,0,0,1,0)\\
\nonumber
9 &=& (0,0,0,0,0,0,0,0,0,1)\quad .
\end{eqnarray}

\subsubsection{Softmax output}

Whereas OHE transforms the training input into a discrete probability distribution, nothing in the neural network will ensure that the data will also come out like that.
A sigmoidal activation function would ensure that each value remains between $0$ and $1$, but a ReLU does not. We therefore need a final layer that ensures each output to be limited to the range $0$ to $1$ \textsl{and} that the sum of all elements to be adding up to one. This is usually achieved using a so-called \textsl{Softmax}\index{Softmax (neural network)} layer. The softmax function is given by:

\begin{equation}
{\sigma (\mathbf {z} )_{j}={\frac {e^{z_{j}}}{\sum _{k=1}^{K}e^{z_{k}}}}} \quad for \quad j=1,\ldots,K
\end{equation}

That is, a vector $z \in \mathbb{R}^K$ will be converted into a $K-$dimensional vector whose $j-$th element is given by the above formula.

So, why not just normalize with the actual values, i.e. using $z_j$ instead of $e^{z_j}$, or, even easier, using $\arg \max_j$ to set the highest value of $z$ to $1$ and leave the rest to zero?
The reason is that each layer needs to remain differentiable for backpropagation to work.
Yet, the ``brutal'' cut-off introduced by the $\arg \max$ function is exactly what we want for the network to optimally match the training input. This is why the exponential function is used. It---literally---exponentially emphasizes larger values over smaller values, making the class with the highest probability stand out.

\section{Objective functions and optimization}

The key idea to train neural networks is to change the network's parameters so that a certain objective function (called loss function), is minimized.
This is usually done by evaluating the gradient of the objective function with respect to the network's parameters. Being differentiable is therefore a key requirement for a useful objective function.
However, the magnitude of the weights can dramatically impact neural network performance and finding this magnitude is entirely dependent on the type of learning problem.

\subsection{Loss functions for regression tasks}

So far, we have considered the so-called \textsl{Mean-squared Error} (MSE):

\begin{equation}
E=\frac{1}{2N}\sum_{i=1}^{N}(\hat{y_i}(w)-y_i)^2\quad ,
\end{equation}

which is the average error over a set of $N$ pairs of predictions $\hat{y}$ that are dependent on the network parameters $w$ and known values $y$---see also \cref{sec:linefitting}. This function is particularly convenient, as the square makes it convex, allowing to find its minimum by following its gradient (``gradient descent'').

MSE is most suited for \textsl{regression} tasks\index{Regression} in which data points are fitted to a model such as a line. Using a sigmoid or other continuous activation function, the error for each class can also be interpreted as a distance from the separating hyperplane, which makes MSE also suitable (but not optimal) for these kind of tasks. An example is illustrated below:

\begin{figure}[htb]
    \centering
    % \includegraphics[width=0.8\columnwidth]{figs/backpropnotation3}
    \def\svgwidth{0.5\textwidth}
    \import{./figs/}{outliers.pdf_tex}
    \caption{A regression problem with an outlier.\label{fig:outliers}}
\end{figure}

From \cref{fig:outliers} it is clear that MSE poorly deals with outliers. If one value deviates largely from the prediction, the quadratic term in MSE will heavily ``punish'' this value. An alternative to MSE is the \textsl{Mean Absolute Error} (MAE):

\begin{equation}
E=\frac{1}{2N}\sum_{i=1}^N\|\hat{y_i}(w)-y_i\|)
\end{equation}

Here, the absolute value ensures that the error is always positive no matter the direction, but large errors are weighted on the same order of magnitude as smaller ones. MAE is therefore better suited if your training set contains outliers.

In practice, a large variety of loss functions have been developed to combine features of both MSE and MAE; in the simplest form of the \textsl{Huber loss}\index{Huber Loss (neural networks)} function, this is achieved via a simple piecewise combination.

\subsection{Loss functions for classification tasks}

Although a classification task can be cast into a regression problem, classifying is more akin to throwing a dice.
Indeed, the output of the Softmax layer is a discrete probability distribution in which each element $y_i=(p_0, \dots, p_c, \dots ,p_N)$ is the probability of an instance $x_i$ to be of class $c$ in $N$ classes total.

We speak of the \textsl{entropy}\index{Entropy (neural networks)} of a probability distribution as the amount of ``variety'' that we expect.
To make an example, a uniform distribution has the highest entropy because there exist a high number of possible outcomes, whereas the one-hot encoded vectors are probability distributions with very low entropy.
The entropy of the distribution of $y_i$ (the training vector that stores the true class $c$ for each instance $x_i$) is given by:

\begin{equation}
H(y_i)=-\sum_{c=1}{N}p_c \log p_c\quad.
\end{equation}

Here, the logarithm can be of basis ten or two. In any case, the entropy function has a couple of interesting properties: first, the logarithm from $0$ (negative infinity) to $1$ is negative (this is why probabilities yield positive values). Second, the logarithm of $1$ is zero, i.e., a distribution with only one element ($p_c=1$) has the lowest possible entropy.
Third, the lower the individual entries for $p_c$ are---for example, in a uniform distribution where $p_c=\frac{1}{N}$, the higher the entropy.

In every dataset, there will always exist a true distribution $P(C=i)$ that the data is distributed according to. By classifying every element in the training set, the neural network also generates its own distribution or ``interpretation'' of the data.
Ideally, in the case of a $100\%$ fit, the neural network will generate (or ``learn'') the exact same distribution as the one that describes the training set. In the worst case, the network will generate a distribution that is completely different. Evaluating a neural network's performance is therefore a matter of comparing two probability distributions.

One way to compare two distributions is via their entropy---a process known as \textsl{cross entropy}:
\begin{equation}
H(\hat{y},y)=-\sum_{i=1}{N}y_i\log \hat{y_i}\quad ,
\end{equation}

with $y_i=p_i$ being the known probability for instance $x$ to be class $i$ and $\hat{y_i}$ being the prediction. As the neural network will never perfectly represent the data, the cross entropy will always be larger than the entropy of the true distribution, that is:
\begin{equation}
H(y)-H(\hat{y},y) \leq 0
\end{equation}

This difference between the entropy of the true distribution and the cross-entropy between the true and the estimated distribution is known as \textsl{Kullback-Leibler Divergence}\index{Kullback-Leibler Divergence}. It is a measure of dissimilarity between two distributions.

\subsection{Binary and Categorical cross-entropy}
In the case where there are only two classes, the \textsl{binary cross-entropy} is calculated as follows:
\begin{equation}
H(\hat{y},y)=-\sum_{i=1}^Ny_i\log(\hat{y_i})=-y_1\log(\hat{y_1})-(1-y_1)\log(1-\hat{y_1})
\end{equation}

As there are only two classes (either true or false), $\hat{y_2}$ directly follows from $1-\hat{y_1}$.
The more general case for $N>2$ is known as \textsl{categorical cross-entropy}.
%
When using one-hot encoding, only class $c$ has probability 1 ($y_c=1$), reducing the cross-entropy to:
\begin{equation}
H(\hat{y},y)=-\log(\hat{y_c})
\end{equation}

with $c$ the true class (the other terms are zero). 
Combined with the softmax activation function the categorical cross entropy therefore computes as
\begin{equation}
H(\hat{y},y) = -\log\left(\frac{e^{\hat{y_c}}}{\sum_{j}^N e^{\hat{y_j}}}\right)
\end{equation}

\section{Convolutional Neural Networks}\label{sec:ann:cnn}

A drawback of the ANN architectures that we have covered so far is that they do not consider the spatial information that might be hidden in a dataset.
For example, as detailed in \cref{chap:vision}, in the context of vision it is important to interpret the value of a certain pixel depending on what can be seen nearby: a blue pixel surrounded by white ones might be an eye, whereas a blue pixel surround by blue ones might be an ocean. In addition to color, neighboring pixels also encode structure. When looking at the MNIST dataset (a collection of hand-drawn numbers from zero to nine), we might for example be looking for crosses (such as the center of an eight), T-shaped junctions (such as in the letter four) or half-circles (like in the letter three), whose number might then serve as features for our neural network.
The SIFT features in \cref{chap:feature_extraction} were a good example of a hand-coded approach to encode such spatial information. We will now see how ANNs can find such features automatically.

If you recall, one way to extract features in image processing is by \textsl{convolving} an image with a \textsl{kernel}\index{Kernel (neural networks)}---see e.g. a convolution with a $3\times3$ and a $7\times7$ kernel in \cref{fig:convolution}.
%
\begin{figure}[htb]
    \centering
    % \includegraphics[width=0.8\columnwidth]{figs/backpropnotation3}
    \def\svgwidth{0.8\textwidth}
    \import{./figs/}{convolution.pdf_tex}
    \caption{Convolution with a $3\times3$ and a $7\times7$ kernel and resulting reduction in image size.\label{fig:convolution}}
\end{figure}
%
During a convolution, the kernel is swept across the input image, summing over a piece-wise multiplication of each element of the kernel with the underlying image pixels (see also \cref{chap:vision}). As all multiplications are summed, such an operation yields only one pixel. As the kernel has to start somewhat inside the image (unless its borders are padded with appropriate values), we are loosing half the width of the kernel on each side. In the example above, a $3\times3$ kernel turns a $28\times28$ input image into a $26\times26$ output image and a $7\times7$ kernel turns it into a $22\times22$ pixel image. Mathematically, the convolution is defined as

\begin{equation}
x(n_1,n_2)*h(n_1,n_2)=\sum_{k_1=-\infty}^{\infty} \sum_{k_2=-\infty}^{\infty} h(k_1,k_2)x(n_1-k_1,n_2-k_2)
\end{equation}

where bounds (here, infinity) need to be chosen so that the kernel starts at the upper left corner of the image and ends at the lower right corner. It is also possible to artificially grow the input image by adding pixels around it, which is known as \textsl{padding}. Note that the resulting output is identical to examples shown in \cref{chap:vision}.

\subsection{From convolutions to 2D neural networks}

When looking at how one individual pixel in the output above gets computed, we assume that the input pixel is labeled $x_{i,j}$ with $i$ the row and $j$ the column of this pixel. We also assume the entries of the convolution kernel to be indexed in a similar way. Using a $3\times3$ kernel, the first pixel of the output is then calculated by:

\begin{eqnarray}
o_{0,0}=&x_{0,0}w_{0,0}+x_{0,1}w_{0,1}+x_{0,2}w_{0,2}\\
\nonumber
		&+x_{1,0}w_{1,0}+x_{1,1}w_{1,1}+x_{1,2}w_{1,2}\\
\nonumber
		&+x_{2,0}w_{2,0}+x_{2,1}w_{2,1}+x_{2,2}w_{2,2}
\end{eqnarray}

This operation is therefore simply computing the dot-product of the value of $9$ pixels with the kernel weights. Adding a bias value and an activation function such as ReLu is therefore identical to adding a hidden layer with nine neurons!

Performing the convolution by moving the convolution kernel with a width of ($2r+1$) across an entire $X\times Y$ image is therefore akin to creating $(X-2r)(Y-2r)$ ``convolutional'' neurons; the resulting structure is called a \textsl{feature map}\index{Feature Map (neural networks)}.
Note that the ``weights'' of the feature map---i.e., the entries of the kernel matrix---are identical for each neuron in the feature map. We can now repeat this step with additional kernels, resulting in multiple feature maps, which then form a \textsl{convolutional layer}\index{Convolutional Layer (neural networks)}.

Importantly, as this structure is very similar to the conventional neural network structure (except for the fact that a large number of weights are identical), the parameters of each kernel can also be trained using backpropagation! See \cref{sec:backpropagation}.

\subsection{Padding and striding}

As mentioned earlier, a convolution of kernel width $2r+1$ reduces the input by $r$ on each side.
If this is not desired (for example, when multiple convolutional layers are used in series), \textsl{padding} can be used to surround the input image with up to $r$ pixels, which results in the output image having the same dimension as the input image. Instead of moving the convolution kernel pixel by pixel, skipping pixels will further reduce the size of the output image. The amount by which the convolution kernel is moved is known as \textsl{stride}. This is illustrated in \cref{fig:stride} for strides of one and three.

\begin{figure}[htb]
    \centering
    % \includegraphics[width=0.8\columnwidth]{figs/backpropnotation3}
    \def\svgwidth{0.8\textwidth}
    \import{./figs/}{stride.pdf_tex}
    \caption{Convolution with $1\times1$ and $3\times3$ stride and resulting output.\label{fig:stride}}
\end{figure}

\subsection{Pooling}

The feature maps that result from convolution each identify specific features that are defined by their kernels.
Though training, it is possible to identify these kernels and specialize them for specific characteristics that are of interest: some might ``fire'' on edges, others on intersections of lines, and others on very specific patterns in the dataset.
Activation functions may be used to further amplify this effect, making a clear distinction between whether a feature is present or not.
However, in most practical applications such features are rather sparse, and whether they exist in a larger area or not might the most salient information. This can be achieved by a \textsl{pooling layer}\index{Pooling (neural networks)}.

A pooling operation applies a window to select the maximum (in which case it is referred to as \textsl{MaxPooling}\index{Max Pooling (neural networks)}) or the average, among many other possible non-linear functions, from a window of a given size. \cref{fig:pooling} shows the result of a MaxPooling layer with pool size of $3\times3$ and stride lengths of $1\times1$ and $3\times3$. Usually, the stride length is the same as the width of the window.

\begin{figure}[htb]
    \centering
    % \includegraphics[width=0.8\columnwidth]{figs/backpropnotation3}
    \def\svgwidth{0.8\textwidth}
    \import{./figs/}{pooling.pdf_tex}
    \caption{Pooling using a pool size of $3\times3$ for different strides and corresponding output.\label{fig:pooling}}
\end{figure}

Although the $max()$ function is not differentiable, MaxPooling can still be used in backpropagation by selectively passing the gradient to only the neuron that has shown to have the maximum activation and setting the gradient of all other neurons to zero. When an averaging pooling function is used, the gradient is divided among all neurons in the pool in equal parts.

\subsection{Flattening}

The first step in previous neural network models has been to flatten a 2D input image into a one-dimensional vector. This has been the precondition to apply a dense layer and has been accomplished during preprocessing.
However, CNNs require multi-dimensional inputs (e.g., 2D images with multiple color channels). Turning a multi-dimensional tensor into a vector is known as \textsl{flattening} and results into simple reordering. For example, an RGB image of dimensionality ($28\times28\times3$) might be turned into $20$ convolutional filters, or $2352$ individual neurons. A flattening layer arranges them again in a single vector.

\subsection{A sample CNN}

\cref{fig:cnn} shows a typical CNN that combines multiple convolutional and pooling layers. The network takes a $28\times28$ image as an input and trains $20$ different $5\times5$ convolution kernels to create $20$ feature maps of $28\times28$ each. This layer is followed by a maxpooling layer that downsamples each feature map by a factor of two. These feature maps are then convolved with $50$ $5\times5$ convolution kernels to create $50$ $14\times14$ feature maps. These will again be downsampled by a maxpooling operation. The resulting $50$ feature maps are then flattened and fed into a hidden layer of 500 neurons, and finally into a SoftMax-activated output layer with $10$ neurons.

\begin{figure}[htb]
\tiny
    \centering
    % \includegraphics[width=0.8\columnwidth]{figs/backpropnotation3}
    \def\svgwidth{\textwidth}
    \import{./figs/}{cnn.pdf_tex}
    \caption{A typical convolutional neural network taking a $28\times28$ input image and reducing it to $10$ classes.\label{fig:cnn}}
\end{figure}

\subsection{Convolutional Networks beyond 2D image data}

Convolution kernels emphasize areas of similarity. This can be readily understood when considering a simple kernel like $[[0,9,0],[0,9,0],[0,9,0]]$ which emphasizes vertical lines but ignores horizontal ones. Training a convolutional network therefore automatically finds regularities in the training set, as well as in the resulting feature map---often generating hierarchical representations by itself. A common example is a convolutional neural network for face detection in which early layers detect low-level features, which then get recombined into noses, ears, mouth and eyes in deeper layers.

Convolutional neural networks are not limited to 2D image data, but can also be applied to 1D time series. Here, the will find distinct patterns, for example peaks in an accelerometer or gyroscope reading, which can then be used collectively to classify complex signals.

\section{Recurrent Neural Networks}

So far, we have only worked with static data. Even if data had a temporal nature, we have simply concatenated inputs and looked at a piece of history all at once. When using a dense network, all inputs are initially of equal importance and it is up to the network to identify salient information. Albeit convolutional layers might help to dictate some sense of order---a 1D convolutional layer might as well be interpreted as detecting a pattern in a timeseries---dense layers focus on the values of individual features, not on the order of information.

For example, it is straightforward to train a neural network controller to transform input data from sensors into motor commands to perform tasks like light following, obstacle avoidance, and wall following such as those described in \cref{chap:taskexecution}; however, such a controller will be purely reactive and not be able to, for example, escape a U-shaped obstacle.

To overcome this limitation, it is useful to introduce a notion of state in a neural network. In this case, the detection of an event such as ``getting stuck'' may be used to modify the network state in some way. This is accomplished using so-called \textsl{recurrent neural networks}\index{Recurrent neural network}\index{RNN}.
A recurrent neural network uses a special kind of neuron, which sums the input $x_t$ at time $t$ with the value of the hidden state $h_{t-1}$ at the previous time step $t-1$ to compute a hidden state $h_t$ at time $t$. Both terms of this sum are weighed by weights W and U. The output of the recurrent layer is the hidden state $h_t$ weighed by a third weight V and ran through a second activation function. The equation below shows the computation of a RNN layer in vector form, passing the hidden states through a softmax activation.

\begin{equation}
h_t = \tanh(Wh_{t-1}+Ux_t)
\end{equation}
\begin{equation}
y_t = softmax(Vh_t)
\end{equation}

This relationship is shown in \cref{fig:rnn}. As an RNN cell is reusing its internal state $h_t$ in the next iteration, a network that looks back $N$ time-steps is modeled as $N$ cells that are laterally connected. As this is how an RNN is actually implemented, the data from all time steps is presented at the same time.

\begin{figure}
\tiny
    \centering
    % \includegraphics[width=0.8\columnwidth]{figs/backpropnotation3}
    \def\svgwidth{\textwidth}
    \import{./figs/}{rnn.pdf_tex}
    \caption{A sample recurrent neural network (left) and its expanded version (right) that is looking back four time steps. \label{fig:rnn}}
\end{figure}

\section*{Take-home lessons}
\begin{itemize}
\item Artificial Neural Networks and the tools associated with them have become a powerful tool to skip modeling a system using first principles, but simply learn its properties from data. As such, they are capable of replacing many of the models discussed in previous chapters, ranging from kinematics to vision, feature detection, and controls.
\item Simple neural networks are capable of both classification and regression akin to techniques described in \cref{chap:feature_extraction}, whereas convolutional networks are capable of filtering and pre-processing techniques such as described in \cref{chap:vision}.
\item When a system is not purely reactive but requires state such as those described in \cref{chap:taskexecution}, recurrent neural networks are needed to implement a notion of memory.
\end{itemize}
\section*{Exercises}\small
\begin{enumerate}
\item Implement the simple perceptron training algorithm and use it to find a separating hyperplane for simple data.
\item Find out how to implement the auto differentiation (or auto gradient) function in your favorite numerical package, e.g. \textsl{NumPy} or \textsl{PyTorch} to automatically calculate the derivative of your loss function.
\item Use a machine learning package of your choice to train a classifier for synthetic images such as the ``Ratslife'' landmarks. If you can, use a real robot to generate appropriate training data.
\item Select a simple 2D target, e.g. a cross on white background, and record images from different distances and angles. Can you train a CNN to predict these two quantities from your image?
\item Select a pre-trained image classifier from your preferred machine learning toolkit and use it as the basis to train your classifier for either landmark recognition or pose recognition. How does using a pre-trained classifier affect learning time and accuracy?
\item What kind of network architecture would you chose to track the robot's location (odometry) based on encoder inputs?
\item Download the ``Robot Execution Failures Data Set'' from the UCI machine learning repository. It contains time-series data from a robot's force-torque sensor as well as whether manipulation was successful. Define a recurrent neural network architecture for this data and train it.
\end{enumerate}\normalsize



\input{chapters/taskexecution}
\input{chapters/mapping}
\input{chapters/pathplanning}
\input{chapters/manipulation}

\part{Uncertainty}
\input{chapters/errorpropagation}
\chapter{Localization}\label{chap:localization}
Robots employ sensors and actuators that are subject to uncertainty. Chapter~\ref{chap:uncertainty} describes how to quantify this uncertainty using probability density functions that associate a probability with each possible outcome of a random process, such as the reading of a sensor or the actual physical change of an actuator. Here, the robot's pose is a compound metric that is of central importance to mobile robotics, and the focus of this chapter.


%One of the most common probability density functions is the Gaussian distribution. It has the shape of a bell and can entirely be described by its mean --- the center of the bell curve --- and its variance, the width of the bell curve.
There are many ways to localize a robot in its environment, and odometry is just one of them. A different possible way to localize a robot in its environment is to extract high-level features (Chapter~\ref{chap:feature_extraction}), such as the distance to a wall from a number of different sensors. %As the underlying measurements are uncertain, these measurements will be subject to uncertainty. How to calculate the uncertainty of a feature from the uncertainty of the sensors that detect this feature, is covered by the error propagation law. The key insight is that the variance of a feature is the weighted sum of all contributing sensors' variances, weighed by their impact on the feature of interest. This impact can be approximated by the derivative of the function that maps a sensor's input to the measurement of the feature.

As we have seen in \cref{chap:uncertainty}, uncertainty keeps propagating without the ability to take corrective measurements. The goals of this chapter are to present mathematical tools and algorithms that will enable you to actually shrink the uncertainty of a measurement by combining it with additional observations.  In particular, this chapter will cover
\begin{itemize}
    \item using landmarks to improve the accuracy of a discrete position estimate (Markov Localization and Bayes Filter),
    \item approximating continuous position estimates (Particle Filter),
    \item using the Extended Kalman Filter to estimate a continuous position estimate.
\end{itemize}

\section{Motivating Example}
Imagine a floor with three doors, two of which are closer together, and the third farther down the corridor (\cref{fig:three_door_example}). Imagine now that your robot is able to detect doors, namely that it is able to tell whether it is in front of a wall or in front of a door. Features like this can serve as a landmark for the robot. Given a map of this simple environment and no information whatsoever about where our robot is located, we can use landmarks to drastically reduce the space of possible locations once the robot has passed one of the doors. One way of representing this belief is to describe the robot's position with three Gaussian distributions, each centered in front of a door and its variance a function of the uncertainty with which the robot can detect a door's center. This is known as a multi-hypothesis belief, since we have a hypothesis stating that the robot can be in front of each door. What happens if the robot continues to move? From the error propagation law we know:
\begin{enumerate}
\item The Gaussians describing the robot's 3 possible locations will move with the robot.
\item The variance of each Gaussian will keep increasing with the distance the robot moves.
\end{enumerate}
What happens if the robot arrives at another door? Given a map of the environment, we can now map the three Gaussian distributions to the location of the three doors. As all three Gaussians will have moved, but the doors are not equally spaced, only some of the peaks will coincide with the location of  a door. Assuming we trust our door detector much more than our odometry estimate, we can now remove all beliefs that do not coincide with a door. Again assuming our door detector can detect the center of a door with some accuracy, our location estimate's uncertainty is now only limited by that of the door detector.


Things are just slightly more complicated if our door detector is also subject to uncertainty: there is a chance that we are in front of a door, but haven't noticed it. Then, it would be a mistake to remove this belief. Instead, we just weigh all beliefs with the probability that there \textsl{could} be a door. Say our door detector detects false-positives with a 10\% chance. Then, there is a 10\% chance to be at any location that is not in front of a door, even if our detector tells us we are in front of a door. Similarly, our detector might detect false-negatives with 20\% chance, telling us that there is no door even though the robot is just in front of it. Thus, we would need to weigh all locations in front of a door with 20\% chance and all locations not in front of a door with 80\% likelihood if our robot tells us there is no door, even if we are indeed in front of one.

\begin{figure}
	\centering
	% \includegraphics[width=\textwidth]{figs/three_door_example}
    \def\svgwidth{\textwidth}
    \import{./figs/}{three_door_example.pdf_tex}
	\caption{A robot localizing itself using a ``door detector'' in a known map. Top: Upon encountering a door, the robot can be in front of any of the three doors. Middle: When driving to the right, the Gaussian distributions representing its location also shift to the right and widen, representing growing uncertainty. Bottom: After detecting the second door, the robot can discard hypotheses that are not in front of the door and gains certainty on its location. 
	\label{fig:three_door_example}}
\end{figure}

\section{Markov Localization}\label{sec:markovloc}
Calculating the probability to be at a certain location given the likelihood of certain observations is the same as any other conditional probability. There is a formal way to describe such situations: Bayes' Rule (\cref{sec:bayesrule})\index{Bayes' rule}:
\begin{equation}
P(A|B)=\frac{P(A)P(B|A)}{P(B)}
\end{equation}

\subsection{Perception Update}
How does this map into a Localization framework? Let's assume event $A$ is equivalent to being at a specific location $loc$. Let's also assume that event $B$ corresponds to the event of seeing a particular feature $feat$. We can now rewrite Bayes' rule to

\begin{equation}
P(loc|feat)=\frac{P(loc)P(feat|loc)}{P(feat)}
\label{eq:bayesloc}
\end{equation}

Rephrasing Bayes' rule in this way, we can calculate the probability to be at location $loc$, given that we see feature $feat$. This is known as \textsl{Perception Update}\index{Perception Update (Markov Localization)}. For example, $loc$ could correspond to door 1, 2 or 3, and $feat$ could be the event of sensing a door. What do we need to know to make use of this equation?
\begin{enumerate}
\item We need to know the prior probability to be at location loc $P(loc)$
\item We need to know the probability of seeing the feature if we were actually at this location $P(feat|loc)$
\item We need the probability of encountering the feature feat $P(feat)$
\end{enumerate}
Let's start with (3), which might be the most confusing part of information we need to collect. It may make more sense to consider $P(feat)=\sum_{x\in locations}P(feat|x)*P(x)$, the probability that we'd see this feature in a given location for every possible location. It is also common to see this term set to $1$, with $P(loc|feat)$ written as being proportional to the numerator of \cref{eq:bayesloc} instead of equals.
%The answer is simple, it is safe to assume the probability to measure the feature to be simply one ($P(feat)=1$).

The prior probability to be at location  $loc$, $P(loc)$, is called the \index{Belief Model} \textsl{belief model}. In the case of the 3-door example, it is the value of the Gaussian distribution underneath the door corresponding to $loc$.

Finally, we need to know the probability $P(feat|loc)$ of seeing the feature $feat$ given that we are at location $loc$. If your sensor was perfect, this probability is simply 1 if the feature exists at this location, or 0 if the feature cannot be observed at this location. If your sensor is not perfect, $P(feat|loc)$ corresponds to the likelihood of the sensor to see the feature if it exists.

The last missing piece involves deciding how to represent possible locations. In the graphical example in \cref{fig:three_door_example} we assumed Gaussian distributions for each possible location. Alternatively, we can discretize the world into a grid and calculate the likelihood of the robot to be in any of its cells. In our 3-door world, it might make sense to choose grid cells that have the width of a door.

\subsection{Action Update}
One of the assumptions in the above thought experiment was that we know with certainty that the robot moved right. We will now more formally study how to treat uncertainty from motion. Recall that odometry input is just another sensor that we assume to have a Gaussian distribution; if our odometer tells us that the robot traveled a meter, it could have traveled a little less or a little more, with decreasing likelihood the further we get from the given measurement. We can therefore calculate the posterior probability of the robot moving from a position $loc'$ to $loc$ given its odometer input $odo$:

\begin{equation}
P(loc'->loc|odo)=P(loc'->loc)P(odo|loc'->loc)/P(odo)
\end{equation}

This is again Bayes' rule. The unconditional probability $P(loc'->loc)$ is the prior probability for the robot to have been at location $loc'$. The term $ P(odo|loc'->loc)$ corresponds to the probability to get odometer reading $odo$ after traveling from a position $loc'$ to $loc$. If getting a reading of the amount $odo$ is reasonable for the distance from $loc'$ to $loc$ this probability is high. If it is unreasonable, for example if the distance is larger than what is physically possible, this probability should be very low.

As the robot's location is uncertain, the real challenge is now that the robot could have potentially been anywhere to start with. We therefore have to calculate the posterior probability $P(loc|odo)$ for all possible positions $loc'$. This can be accomplished by summing over all possible locations:
\begin{equation}
P(loc|odo)=\sum_{loc'}P(loc'->loc)P(odo|loc'->loc)
\end{equation}
In other words, the law of total probability requires us to consider all possible locations the robot could have ever been at. This step is known as the \textsl{Action Update}\index{Action Update (Markov Localization)}. In practice we don't need to calculate this for all possible locations, but only those that are technically feasible given the maximum speed of the robot. We note also that the sum notation technically corresponds to a convolution (\cref{sec:convolution}) of the probability distribution of the robot's location in the environment with the robot's odometry error probability distribution.

\subsection{Example: Markov Localization on a Topological Map}
We have now learned two methods to update the belief distribution of where the robot could be in the environment. First, a robot can use external landmarks to update its position. This is known as the \textsl{perception update} and relies on exterioception. Second, a robot can observe its internal sensors. This is an instance of an \textsl{action update} and relies on proprioception. The combination of action and perception updates is known as \textsl{Markov Localization}\index{Markov Localization}. You can think about the action update as increasing the uncertainty of the robot's position and the perception update as shrinking it. (You can also think about the action update as a discrete version of the error propagation model.) %Also here we are using the robotics kinematic model and the noise model of the odometer to calculate $ P(odo|loc'->loc)$.

%\paragraph{Example 1: Topological Map}
To illustrate this, we now describe one of the first successful real robot systems that employed Markov Localization in an office environment. The experiment is described in more detail in a 1995 article of AI Magazine\cite{nourbakhsh1995dervish}. The office environment consisted of two rooms and a corridor that can be modeled by a topological map\index{Topological Map} (\cref{fig:dervish_example}). In a topological map, areas that the robot can be in are modeled as vertices, and navigable connections between them are modeled as edges of a graph. The location of the robot can now be represented as a probability distribution over the vertices of  this graph.

\begin{figure}
	\centering
	% \includegraphics[width=\textwidth]{figs/dervish_example}
    \def\svgwidth{\textwidth}
    \import{./figs/}{dervish_example.pdf_tex}
	\caption{An office environment consisting of two rooms connected by a hallway. A topological map is super-imposed.
	\label{fig:dervish_example}}
\end{figure}

The robot has the following sensing abilities:
\begin{itemize}
    \item It can detect a closed door to its left or right.
    \item It can detect an open door to its left or right.
    \item It can detect whether it is an open hallway.
\end{itemize}

Unfortunately, the robot's sensors are not at all reliable. The researchers have experimentally found the probabilities to obtain a certain sensor response for specific physical positions using their robot in their environment. These values are provided in \cref{tab:dervish_example}.

\begin{table}
\footnotesize
\begin{tabular}{lccccc}
 	& Wall	& Closed dr & Open dr	& Open hwy & Foyer\\
\hline
Nothing detected	& 70\%	& 40\%&	5\%	& 0.1\% & 30\%\\
Closed door detected & 30\% &	60\%& 0\% &0\%	& 5\%\\
Open door detected & 0\%	& 0\%&	90\% & 10\% & 15\%\\
Open hallway detected & 0\% &	 0\%&	0.1\% & 90\% &50\%\\
\hline
\end{tabular}
\normalsize
\caption{Conditional probabilities of the Dervish robot detecting certain features in the Stanford laboratory.\label{tab:dervish_example}}
\end{table}

For example, the success rate to detect a closed door is only 60\%, whereas a foyer looks like an open door in 15\% of the trials. This data corresponds to the conditional probability to detect a certain feature given a certain location.

Consider now the following initial belief state distribution: $p(`1-2')=0.8$ and $p(`2-3')=0.2$. Here, `$1-2$' and `$2-3$' refer to the positions on the topological map in \cref{fig:dervish_example}. For this domain, we are told with certainty that the robot faces east. The robot now drives for a while until it reports ``open hallway on its left and open door on its right''. This actually corresponds to location 2, but the robot can in fact be anywhere. For example there is a 10\% chance that the open door is in fact an open hallway, i.e.\ the robot is really at position 4. How can we calculate the new probability distribution of the robot's location? Here are the possible trajectories that could happen:

The robot could move from $2-3$ to $3$, $3-4$ and finally $4$. We have chosen this sequence as the probability to detect an open door on its right is zero for $3$ and $3-4$, which leaves position $4$ as the only option if the robot has started at $2-3$. In order for this hypothesis to be true, the following events need to have happened, their probabilities are given in parentheses:
\begin{enumerate}
    \item The robot must have started at $2-3$ (20\%)
    \item Not have seen the open door at the left of $3$ (5\%) and not have seen the wall at the right (70\%)
    \item Not have seen the wall to its left (70\%) and not have seen the wall to its right at node $3-4$ (70\%)
    \item Correctly identify the open hallway to its left (90\%) and mistake the open hallway to its right for an open door (10\%)
\end{enumerate}
Together, the likelihood that the robot got from position $2-3$ to position $4$ is therefore given by $0.2\times 0.05 \times 0.7 \times 0.7 \times 0.7 \times 0.9 \times0.1=0.03\%$, that is very unlikely.

The robot could also move from $1-2$ to $2$, $2-3$, $3$, $3-4$ or $4$. We can evaluate these hypotheses in a similar way:
\begin{itemize}
    \item The chance that it correctly detects the open hallway and door at position $2$ is $0.9 \times 0.9$, so the chance to be at position $2$, having started at $1-2$, is $0.8 \times 0.9 \times 0.9=64\%$.
    \item The robot cannot have ended up at position $2-3$, $3$, and $3-4$ because the chance of seeing an open door instead of a wall on the right side is zero in all these cases.
    \item In order to reach position $4$, the robot must have started at $1-2$ has a chance of $0.8$. The robot must not have seen the hallway on its left and the open door to its right when passing position $2$. The probability for this is $0.001 \times 0.05$. The robot must then have detected nothing at $2-3$ ($0.7 \times 0.7$), nothing at $3$ ($0.05 \times 0.7$), nothing at $3-4$ ($0.7 \times 0.7$), and finally mistaken the hallway on its right for an open door at position $4$ ($0.9 \times 0.1$). Multiplied together, this outcome is very unlikely.
\end{itemize}

Given this information, we can now calculate the posterior probability to be at a certain location on the topological map by adding up the probabilities for every possible path to get there.


\section{The Bayes Filter}\index{Bayes Filter}
We have seen how sensor measurements can be formally incorporated into a position estimate using Bayes' rule, which relates the likelihood of being at a certain position given that the robot sees a certain feature to the likelihood of the robot seeing this feature if it were really at the hypothetical location. We have also seen how the robot can use its sensor model to relate its observation with possible positions. 
Its real location is likely to be somewhere between its original belief (based on error propagation) and where the sensor tells it that it is. We will now provide an algorithm for localizing a robot through a multi-hypothesis, iterative process that does not depend on a particular class of motion or sensor model (e.g., the Gaussian noise models used by Kalman Filters).

To formalize our terms and notation, we will describe our robot's motion model as the distribution given by $P(x'|x,u)$, that is, the probability of being in a particular state $x'$ given that we started in state $x$ and executed action $u$. We can describe our sensor model as being characterized by the distribution given by $P(z|x)$, namely the probability that we would see sensor observation $z$ if we were in state $x$. This is not limited to discrete locations as in the previous sensor, but could also be the likelihood of an ultrasound sensor detecting a wall a certain distance. Typically, this will require some discretization of the environment, such as a grid. Finally, we will define the probability of being in a particular state $x$ as $P(x)$. 

Our goal with the Bayes filter will be to estimate our robot's state over time ($x_t$, where $t$ indicates timestep) given a history of actions and observations (sensor measurements). To do so, we will compute the posterior probability of our state estimate, also known as \textsl{belief}\index{Belief}, using this history. We define the belief that our robot is in state $x$ at time $t$ given a history of actions ($u_1,...u_t$) and sensor measurements ($z_1,...,z_t$) as:
$$
Bel(x_t) = P(x_t|u_1,z_1,u_2,z_2,...,u_t,z_t)
$$

By leveraging the Markov assumption, that our current state only depends on our previous state $x_{t-1}$ and action $u_t$, we can greatly simplify the computation required. 
$$
P(x_t|x_{0:t-1}, z_{1:t-1}, u_{1:t}) = P(x_t|x_{t-1},u_t)
$$
For example, if we wanted to calculate the probability of an observation $z_t$, we know that the only term that actually matters is the robot's current state (since the other terms don't affect what sensor readings we'd expect to get).
$$
P(z_t|x_{0:t}, z_{1:t-1}, u_{1:t}) = P(z_t|x_t)
$$

We will now derive a recursive definition for belief that makes iteratively computing state belief over a time history of actions and observations tractable. Beginning with our initial definition of belief, we will apply Bayes rule, the Markov property, the law of total probability, and recursion to achieve our goal. We will use $c$ to denote the normalizing constant (from the denominator of Bayes rule), which is the same for all possible $x_t$.
\begin{align}
	Bel(x_t) = {}&  P(x_t|u_1,z_1,...,u_t,z_t)\\
	Bel(x_t) = {}& c * P(z_t|x_t,u_1,z_1,...,u_t,z_t)*P(x_t|u_1,z_1,...,u_t)\\
	Bel(x_t) = {}& c * P(z_t|x_t)*P(x_t|u_1,z_1,...,u_t)\\
	\begin{split}
	Bel(x_t) = {}& c * P(z_t|x_t) *\\&  \sum_{x_{t-1}\in X}P(x_t|u_1,z_1,...,u_t,x_{t-1})*P(x_{t-1}|u_1,z_1,...,z_{t-1},u_t)
	\end{split}\\
	Bel(x_t) = {}& c * P(z_t|x_t)*  \sum_{x_{t-1}\in X}{P(x_t|u_t,x_{t-1})*P(x_{t-1}|u_1,z_1,...,z_{t-1})}\\
	Bel(x_t) = {}& c * P(z_t|x_t) * \sum_{x_{t-1}\in X}{P(x_t|u_t,x_{t-1})*Bel(x_{t-1})}
\end{align}

This final equation is remarkable because it allows us to perform a belief update for a given state by incorporating a sensor measurement and/or a motion prediction based on an action we took. With this formulation, we can define an algorithm for belief updates that takes our current belief, an array of action and observation data, and the set of states that comprise the state space as inputs, returning an updated belief that incorporates this information.

\begin{Verbatim}[commandchars=\\\{\}, codes={\catcode`$=3\catcode`^=7\catcode`_=8}]
BayesFilter(Belief Bel, Data d, Set of States X):
  while d is not empty:
    c = 0
    if (d[0] is a sensor measurement):
      z = d.pop(0)
      for all $x \in X$:
        Bel'(x) = P(z|x)Bel(x)
        c += Bel'(x)
      for all $x \in X$:
        Bel'(x) = $c^{-1}$*Bel'(x)
    elif (d[0] is an action):
      u = d.pop(0)
      for all $x \in X$:
        Bel'(x) = $\sum_{x_{t-1}}P(x|u,x_{t-1})*$Bel$(x_{t-1})$
    Bel = Bel'
  return Bel
\end{Verbatim}

This powerful idea of iteratively incorporating sensor measurements and motion predictions underpins an entire family of state estimation methods. In the sections that follow, we will extend this concept to be applicable in contexts that we often find robots: infinitely large, continuous state spaces that we cannot exhaustively iterate over.

\subsection{Example: Bayes filter on a grid}

\begin{figure}
	\centering
	% \includegraphics[width=\textwidth]{figs/markov_grid_example}
    \def\svgwidth{\textwidth}
    \import{./figs/}{markov_grid_example.pdf_tex}
	\caption{Markov localization on a grid. The left column shows the likelihood to be in a specific cell as grey value (dark colors correspond to high likelihoods). The right column shows the actual robot location. Arrows indicate previous motion. Initially, the position of the robot is unknown, but recorded upwards motion makes positions at the top of the map more likely. After the robot has encountered a wall, positions away from walls become unlikely. After rightwards and down motions, the possible positions have shrunk to a small area.}
	\label{fig:markov_grid_example}
\end{figure}

Instead of using a coarse topological map, we can also model the environment as a fine-grained grid. Each cell is marked with a probability corresponding to the likelihood of the robot being at this exact location (\cref{fig:markov_grid_example}). We assume that the robot is able to detect walls with some certainty, perhaps with a short-range ultrasonic sensor on the front, back, and sides of the robot. The images in the right column show the actual location of the robot, while the left column shows the probability of the robot being in each grid cell. Initially, the robot does not see a wall and therefore could be almost anywhere. The robot now moves northwards. The action update now propagates the probability of the robot being somewhere north. As soon as the robot encounters the wall, the perception update bumps up the likelihood to be higher in grid cells near walls. As there is some uncertainty associated with the wall detector, the robot will not only have likelihood directly at the wall, but also at other locations --- with decreasing probability --- close by to walls. As the action update involved continuous motion to the north, the likelihood that the robot is close to the south wall is almost zero. The robot then performs a right turn and travels along the wall in the clockwise direction. As soon as it hits the east wall, it is almost certain about its position, which then again decreases as the robot continues to travel.


%\section{The Kalman Filter}
%To overcome the limitations of the Bayes filter, we will consider modeling our belief distribution using a family of probability distributions that can be parameterized, as opposed to defining our belief distribution in a discrete point-wise manner. To motivate our choice of distribution family, we turn to the Central Limit Theorem\index{Central Limit Theorem}, which stipulates that random variables that are determined by the sum of lots of small effects are normally distributed. In other words, given an infinite sequence of independent random variables $X_1, X_2, ...$, with $E[X_i]=\mu$ (mean) and $E[X_i-\mu]=\sigma^2$ (variance). Let the value $Z_n$ be defined for $n$ random variables as follows:
%$$
%Z_n = \frac{(X_1+...+X_n)-n\mu}{\sigma\sqrt{n}}
%$$
%As $n\rightarrow \infty$, $Z_n$ is distributed according to the zero-mean, unit-variance Gaussian 
%$\mathcal{N}(0,1)$.\\


\section{Particle Filter}
Although grid-based Markov Localization can provide compelling results, it can be computationally very expensive, in particular when the environment is large and the resolution of the grid is small. This is in part due to the fact that we need to carry the probability to be at a certain location forward for every cell on the grid, regardless of how small this probability is. An elegant solution to this problem is the particle filter. It works as follows:
\begin{enumerate}
\item Represent the robot's position by $N$ particles that are randomly distributed around its estimated initial position. For this, we can either use one or more Gaussian distributions around the initial estimate(s) of where the robot is, or choose an uniform distribution (\cref{fig:particlefilter_example}).
\item Every time the robot moves, we will move each particle in the exact same way, but add noise to each movement much like we would observe on the real robot. Without a perception update, the particles will spread apart farther and farther.
\item Upon a perception event, we evaluate every single particle using our sensor model. What would the likelihood be to have a perception event such as we observed at this location? We can then use Bayes' rule to update each particle's position.
\item Once in a while or during perception events that render certain particles infeasible, particles that have a probability that is too low can be deleted, while those with the highest probability can be replicated.
\end{enumerate}

\begin{figure}[!htb]
	\centering
	% \includegraphics[width=\textwidth]{figs/particlefilter_example}
    \def\svgwidth{0.92\textwidth}
    \import{./figs/}{particlefilter_example.pdf_tex}
	\caption{Particle filter example. Possible positions and orientations of the robot are initially uniformly distributed. Particles move based on the robot's motion model. Particles that would require the robot to move through a wall in absence of a wall perception event are deleted (stars). After a perception event, particles too far from a wall become too unlikely and are resampled to be in the vicinity of a wall. Eventually, the particle filter converges.
	\label{fig:particlefilter_example}}
\end{figure}


\paragraph{Observation}
Let us now assume that we can detect line features $ \boldsymbol{z}_{k,i}=(\alpha_i,r_i)^T$, where $ \alpha$ and $ r$ are the angle and distance of the line from the coordinate system of the robot. These line features are subject to variances $ \sigma_{\alpha,i}$ and $ \sigma_{r,i}$. See the line detection section for a derivation of how angle and distance as well as their variance can be calculated from a laser scanner. The observation is a 2x1 matrix.

\paragraph{Measurement Update}
We assume that we can uniquely identify the lines we are seeing and retrieve their real position from a map that we have been given in advance. This is much easier for unique features, but can also be done for lines by assuming that our error is small enough and we therefore can search through our map and pick the closest lines. As features are stored in global coordinates, we need to transform them into how the robot would see them. In practice this is nothing but a list of lines, each with an angle and a distance, but this time with respect to the origin of the global coordinate system. Transforming them into robot coordinates is straightforward. With  $ \hat{\boldsymbol{x}}_{k}=(x_{k},y_{k},\theta_k)^T$ and $ m_i=(\alpha_i,r_i)$ the corresponding entry from the map, we can write

\begin{equation} h(\hat{\boldsymbol{x}}_{k|k-1})=\left[\begin{array}{c}\alpha_{k,i}\\r_{k,i}\end{array}\right]=h(\boldsymbol{x},m_i)=\left[\begin{array}{c}\alpha_i-\theta\\r_i-(x cos(\alpha_i)+y sin(\alpha_i)\end{array}\right]
\end{equation}

and calculate its Jacobian $ \boldsymbol{H}_{k}$ as the partial derivatives of $ \alpha$ to $ x,y,\theta$ in the first row, and the partial derivatives of $ r$ in the second. How to calculate $ h()$ to predict the radius at which the robot should see the feature with radius $ r_i$ from the map is illustrated in the figure below.


%\begin{mdframed}
%Example on how to predict the distance to a feature the robot would see given its estimated position and its known location from a map.
%\end{mdframed}

\paragraph{Matching}
We are now equipped with a measurement $ \boldsymbol{z}_k$ and a prediction $ h(\hat{\boldsymbol{x}}_{k|k-1})$ based on all features stored in our map. We can now calculate the innovation
\begin{equation}
\tilde{\boldsymbol{y}}_{k}=\boldsymbol{z}_{k}-h(\hat{\boldsymbol{x}}_{k|k-1})
\end{equation}

which is simply the difference between each feature that we can see and those that we predict from the map. The innovation is again a 2x1 matrix.

A major strength of particle filters is that they are non-parametric estimators of arbitrary probability distributions, and thus are able to accommodate non-linear functions that Kalman Filters cannot. However, this is not the only algorithm available for utilizing non-linear motion and sensor models for state estimation. We now introduce a modification to the Kalman Filter enabling the use of non-linear models.


\section{Extended Kalman Filter}\label{sec:EKF}
In contrast to the linear models required of the Kalman Filter, in the Extended Kalman Filter the state transition and observation models do not need to be linear functions of the state but may instead be any function so long as it's differentiable. The action prediction step looks as follows:
\begin{equation}
\hat{\boldsymbol{x}}_{k'|k-1} = f(\hat{\boldsymbol{x}}_{k-1}, \boldsymbol{u}_{k-1})
\end{equation}
Here $ f()$ is a function of the previous state $ \boldsymbol{x}_{k-1}$ and control input $ \boldsymbol{u}_{k-1}$. A good example for such an equation is the odometry update we are already familiar with. Here, $ f()$ is a function describing the forward kinematics of the robot, $ \boldsymbol{x}_k$ its position and $ \boldsymbol{u}_k$ the wheel-speed we set.

Sticking with our well known example, we can also calculate the covariance matrix of the robot position

\begin{equation}
\boldsymbol{P}_{k'|k-1} = \nabla_{x,y,\theta}f \boldsymbol{P}_{k-1|k-1}\nabla_{x,y,\theta}f^T + \nabla_{\Delta_{r,l}}f\boldsymbol{Q}_{k-1}\nabla_{\Delta_{r,l}}f^T
\end{equation}

where $ \boldsymbol{Q}_k$ is the covariance matrix of the process noise (wheel-slip) and the Jacobian matrices of the forward kinematic equations $ f()$ with respect to the robot's position (indicated by the index $ x,y,\theta$) and with respect to the wheel-slip of the left and right wheel.

The perception update step now looks as follows:

\begin{eqnarray}
\hat{\boldsymbol{x}}_{k|k'} &=& \hat{\boldsymbol{x}}_{k'|k-1} + \boldsymbol{K}_{k'}\tilde{\boldsymbol{y}}_{k'}\\
\boldsymbol{P}_{k|k'} &=& (I - \boldsymbol{K}_{k'} {\boldsymbol{H}_{k'}}) \boldsymbol{P}_{k'|k-1}
\end{eqnarray}

We are calculating everything twice: once we update from $ k-1$ to an intermediate result $ k'$ during the action update using our motion model, we obtain the final result after performing the perception update where we go from $ k'$ to $ k$.

We need to calculate three additional variables:
\begin{enumerate}
\item The innovation $ \tilde{\boldsymbol{y}}_{k}=\boldsymbol{z}_{k}-h(\hat{\boldsymbol{x}}_{k|k-1})$
\item The covariance of the innovation $\boldsymbol{S}_{k}={\boldsymbol{H}_{k}}\boldsymbol{P}_{k|k-1}{\boldsymbol{H}_{k}^\top}+\boldsymbol{R}_{k}$
\item The (near-optimal)  Kalman gain $ \boldsymbol{K}_{k}=\boldsymbol{P}_{k|k-1}{\boldsymbol{H}_{k}^\top}\boldsymbol{S}_{k}^{-1}$
\end{enumerate}
Here $ h()$ is the observation model, $ \boldsymbol{H}$ its Jacobian, and $ \boldsymbol{R}_{k}$ the covariance matrix of the observation noise. How these equations are derived is involved (and is one of the fundamental results in control theory), but the idea is the same as introduced above: we wish to minimize the error of the prediction.

\subsection{Odometry using the Kalman Filter}
We will show how a mobile robot equipped with a laser scanner that has a map of the environment can correct its position estimate by relying on unreliable odometry and unreliable sensing, in an optimal way.
Whereas the update step is equivalent to forward kinematics and error propagation that we have seen before, the observation model and calculating the ``innovation'' require additional steps to perform odometry.

\paragraph{1. Prediction}
We assume for now that the reader is familiar with calculating $ \hat{\boldsymbol{x}}_{k'|k-1}=f(x,y,\theta)^T$ and its variance $ \boldsymbol{P}_{k'|k-1}$. Here, $ \boldsymbol{Q}_{k-1}$, the covariance matrix of the wheel-slip error,  is given by
\begin{equation}
\boldsymbol{Q}_{k-1}=\left[\begin{array}{cc}k_r|\Delta s_r & 0\\0 & k_l|\Delta s_l|\end{array}\right]
\end{equation}
where $\Delta s_l$ and $\Delta s_r$ is the wheel movement of the left and right wheel and $k_l$ and $k_r$ are constants. Refer to the odometry lab for detailed derivations of these calculations and how to estimate $ k_r$ and $ k_l$.  The state vector $ \boldsymbol{\hat{x}_{k'|k-1}}$ is a 3$\times$1 vector, the covariance matrix $ \boldsymbol{P_{k'|k-1}}$ is a 3$\times$3 matrix, and $ \nabla_{\Delta_{r,l}}$ that is used during error propagation is a 3$\times$2 matrix. See the error propagation section for details on how to calculate $ \nabla_{\Delta_{r,l}}$.

%% crh: I don't know who's todo this was, but it wasn't mine... commenting for now.

%\td{EDIT THIS FROM MAPPING TO LOCALIZATION ONLY}
%i%\paragraph{2. Observation}
%Let us now assume that we can detect the motion of the robot through an optical flow sensor.
\paragraph{2. Observation}
Let us now assume that we can detect line features $ \boldsymbol{z}_{k,i}=(\alpha_i,r_i)^T$, where $ \alpha$ and $ r$ are the angle and distance of the line from the coordinate system of the robot. These line features are subject to variances $ \sigma_{\alpha,i}$ and $ \sigma_{r,i}$, which make up the diagonal of $ \boldsymbol{R}_{k}$. See the line detection section for a derivation of how angle and distance as well as their variance can be calculated from a laser scanner. The observation is a 2$\times$1 matrix (representing angle and distance).

\paragraph{3. Measurement Update}
We assume that we can uniquely identify the lines we are seeing and retrieve their real position from a map. This is much easier for unique features, but can also be done for lines by assuming that our error is small enough and that we can search through our map and pick the closest lines. As features are stored in global coordinates, we need to transform them into how the robot would see them. In practice this is nothing but a list of lines specified with respect to the origin of the global coordinate system, each with an angle and a distance. Transforming them into robot coordinates is straightforward. With  $ \hat{\boldsymbol{x}}_{k}=(x_{k},y_{k},\theta_k)^T$ and $ m_i=(\alpha_i,r_i)$ the corresponding entry from the map, we can write

\begin{equation} h(\hat{\boldsymbol{x}}_{k|k-1})=\left[\begin{array}{c}\alpha_{k,i}\\r_{k,i}\end{array}\right]=h(\boldsymbol{x},m_i)=\left[\begin{array}{c}\alpha_i-\theta\\r_i-(x cos(\alpha_i)+y sin(\alpha_i)\end{array}\right]
\end{equation}

and calculate its Jacobian $ \boldsymbol{H}_{k}$ as the partial derivatives of $ \alpha$ to $ x,y,\theta$ in the first row, and the partial derivatives of $ r$ in the second. %How to calculate $ h()$ to predict the radius at which the robot should see the feature with radius $ r_i$ from the map is illustrated in the figure below.


%\begin{mdframed}
%Example on how to predict the distance to a feature the robot would see given its estimated position and its known location from a map.
%\end{mdframed}

\paragraph{4. Matching}
We are now equipped with a measurement $ \boldsymbol{z}_k$ and a prediction $ h(\hat{\boldsymbol{x}}_{k|k-1})$ based on all features stored in our map. We can now calculate the innovation
\begin{equation}
\tilde{\boldsymbol{y}}_{k}=\boldsymbol{z}_{k}-h(\hat{\boldsymbol{x}}_{k|k-1})
\end{equation}

which is simply the difference between each feature that we can actually see (our sensor measurement) and the measurement values that we would expect if making a prediction using the map (not using our sensors). The innovation is again a 2x1 matrix.

\paragraph{5. Estimation}
We now have all the ingredients to perform the perception update step of the Kalman filter:

\begin{eqnarray}
\hat{\boldsymbol{x}}_{k|k'} &=& \hat{\boldsymbol{x}}_{k'|k-1} + \boldsymbol{K}_{k'}\tilde{\boldsymbol{y}}_{k'}\\
\boldsymbol{P}_{k|k'} &=& (I - \boldsymbol{K}_{k'} {\boldsymbol{H}_{k'}}) \boldsymbol{P}_{k'|k-1}
\end{eqnarray}

It will provide us with an update of our position that fuses our odometry input and the information that we can extract from features in the environment in a way that takes into account their variances. That is, if the variance of your previous position is high (because you have no idea where you are), but the variance of your measurement is low (maybe from a GPS or a highly recognizable symbol on the wall), the Kalman filter will put more emphasis on your sensor. If your sensors are poor (maybe because you cannot tell different lines/walls apart), more emphasis will be placed on the odometry.

As the state vector is a 3$\times$1 vector and the innovation a 2$\times$1 matrix, the Kalman gain must be a 3$\times$2 matrix. This can also be seen when looking at the covariance matrix that must come out as a 3x3 matrix, and knowing that the Jacobian of the observation function is a 2$\times$3 matrix. We can now calculate the covariance of the innovation and the Kalman gain using

\begin{eqnarray}
\boldsymbol{S}_{k}&=&{\boldsymbol{H}_{k}}\boldsymbol{P}_{k|k-1}{\boldsymbol{H}_{k}^\top}+\boldsymbol{R}_{k}\\
\boldsymbol{K}_{k}&=&\boldsymbol{P}_{k|k-1}{\boldsymbol{H}_{k}^\top}\boldsymbol{S}_{k}^{-1}
\end{eqnarray}



\section{Summary: Probabilistic Map based localization}\label{sec:pmbl}
In order to localize a robot using a map, we need to perform the following steps:
\begin{enumerate}
\item Calculate an estimate of our new position using the forward kinematics and knowledge of the wheel-speeds that we sent to the robot until the robot encounters some uniquely identifiable feature.
\item Calculate the relative position of the feature (a wall, a landmark or beacon) to the robot.
\item Use knowledge of where the feature is located in global coordinates to predict what the robot should see.
\item Calculate the difference between what the robot actually sees and what it believes it should see (e.g.\ using a Kalman filter).
\item Use the result from (4) to update its belief by weighing each observation against its variance.
\end{enumerate}

Steps 1-2 are based on the sections on ``Forward Kinematics'' and ``Line detection''. Step 3 uses again simple forward kinematics to calculate the position of a feature stored in global coordinates in a map in robot coordinates. Step 4 is a simple subtraction of what the sensor sees and what the map says. Step 5 may induce the Kalman filter, or an error minimization constraint. %, which we will describe here. Its derivation is involved, but its intuition is simple: why just averaging between where I think I am and what my sensors tell me, if my sensors are much more reliable and should carry much higher weight?


\section*{Take home lessons}
\begin{itemize}
\item If the robot has no additional sensors and its odometry is noisy, error propagation will lead to ever increasing uncertainty of a robot's position regardless of using Markov localization or the Kalman filter.
\item Once the robot is able to sense features with known locations, Bayes' rule can be used to update the posterior probability of a possible position. The key insight is that the conditional probability to be at a certain position given a certain observation can be inferred from the likelihood to actually make this observation given a certain position.
\item A complete solution that performs this process for discrete locations is known as Markov Localization.
\item The Extended Kalman Filter is the optimal way to fuse observations of different random variables that are Gaussian distributed.
%It is derived by minimizing the least-square error between prediction and real value.
\item Possible random variables could be the estimate of your robot position from odometry and observations of static beacons with known location (but uncertain sensing) in the environment.
\item In order to take advantage of the approach, you will need differentiable functions that relate measurements to state variables as well as an estimate of the covariance matrix of your sensors.
\item An approximation that combines benefits of Markov Localization (multiple hypothesis) and the Kalman filter (continuous representation of position estimates) is the Particle filter.
\end{itemize}

\section*{Exercises}\small
\begin{enumerate}
\item Assume that the ceiling is equipped with infrared markers that the robot can identify with some certainty. Your task is to develop a probabilistic localization scheme, and you would like to calculate the probability $p(marker|reading)$ to be close to a certain marker given a certain sensing reading and information about how the robot has moved.
\begin{enumerate}
\item Derive an expression for $p(marker|reading)$ assuming that you have an estimate of the probability to correctly identify a marker $p(reading|marker)$ and the probability $p(marker)$ of being underneath a specific marker.
\item Now assume that the likelihood that you are reading a marker correctly is 90\%, that you get a wrong reading is 10\%, and that you do not see a marker when passing right underneath it is 50\%. Consider a narrow corridor that is equipped with 4 markers. You know with certainty that you started from the entry closest to marker 1 and move right in a straight line. The first reading you get is ``Marker 3''. Calculate the probability to be indeed underneath marker 3.
\item Could the robot also possibly be underneath marker 4?
\end{enumerate}
\end{enumerate}\normalsize

\input{chapters/SLAM}

\part{Appendices}
\appendix

\input{chapters/trigonometry}
\input{chapters/linearalgebra}
\input{chapters/statistics}
\input{chapters/backpropagation}

\input{chapters/paperwriting}
\input{chapters/samplecurricula}


\bibliographystyle{agsm}
\bibliography{robotics}

\printindex

\end{document}
